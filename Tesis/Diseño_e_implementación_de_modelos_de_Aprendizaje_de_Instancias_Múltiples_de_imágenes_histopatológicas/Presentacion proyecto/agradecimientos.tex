Expresamos nuestro agradecimiento al Dr. Julian Gil González, director de este proyecto, cuya entrega y compromiso incondicional fueron los cimientos sobre los cuales se construyó esta investigación. Más allá de su orientación experta y técnica, su seguimiento constante y su disposición para el diálogo constructivo fueron fundamentales para navegar la complejidad de este trabajo. Su guía no solo nos permitió alcanzar los objetivos académicos planteados, sino que se convirtió en una fuente de inspiración y resiliencia para superar los desafíos inherentes al aprendizaje profundo y la ciencia de datos, fomentando en nosotros un pensamiento crítico y riguroso.\\

Asimismo, extendemos nuestra gratitud al cuerpo docente de la Maestría en Ciencia de Datos de la Pontificia Universidad Javeriana Cali. Cada cátedra, debate y lección compartida en las aulas ha contribuido a nuestra formación integral. Su rigor académico y la pasión por transmitir el conocimiento han sido los pilares que nos permitieron consolidar las competencias técnicas y analíticas necesarias para alcanzar este título profesional. Gracias por exigirnos excelencia y por proporcionarnos las herramientas para enfrentar los retos del mundo tecnológico actual.\\

Finalmente, dedicamos este logro con amor y gratitud infinita a nuestras familias. Ustedes han sido nuestro refugio y fortaleza a lo largo de este exigente camino; gracias por su apoyo inagotable, por la sabiduría de sus consejos y por ser ese soporte emocional incondicional que nos impulsó a perseverar en los momentos de incertidumbre. Su presencia, sacrificio y confianza ciega en nuestras capacidades han sido el motor fundamental y la razón de ser de este esfuerzo. La culminación exitosa de este trabajo de grado es, en esencia, un triunfo compartido con ustedes.

