Este trabajo de grado presenta el diseño e implementación de un pipeline reproducible basado en Aprendizaje de Instancias Múltiples (MIL) para la clasificación débilmente supervisada de imágenes de lámina completa (WSI) de cáncer de próstata. El proyecto aborda la problemática de la variabilidad interobservador y la escasez de anotaciones locales mediante un enfoque que utiliza extractores de características robustos (ResNet-50) y mecanismos de atención para la agregación de parches. La metodología integró preprocesamiento en el espacio de color HSV, extracción de embeddings y entrenamiento con etiquetas a nivel de lámina sobre la base de datos pública SICAPv2, empleando una validación cruzada estratificada por paciente (GroupKFold) para garantizar el rigor científico.\newline

Los resultados alcanzaron un \textbf{F1-score promedio de 0.88}, un \textbf{PR AUC de 0.917} y un \textbf{Coeficiente Kappa de 0.887}, demostrando un acuerdo "casi perfecto" con el estándar de oro médico. El sistema no solo clasifica eficazmente el tejido tumoral, sino que aporta salidas explicables mediante mapas de atención que localizan estructuras glandulares críticas (como glándulas cribiformes), alineándose con los criterios de la escala de Gleason. Se entrega un entorno versionado y reproducible que facilita la validación clínica y establece una línea base sólida para la patología digital asistida.

\medskip
\textbf{Palabras Claves:} Aprendizaje de instancias múltiples (MIL); Imágenes de lámina completa (WSI); Redes neuronales convolucionales; Histopatología; Cáncer de próstata; Inteligencia Artificial Explicable (XAI).