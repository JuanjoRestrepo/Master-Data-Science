This project presents the design and implementation of a reproducible pipeline based on Multiple Instance Learning (MIL) for the weakly supervised classification of prostate cancer Whole Slide Images (WSI). The study addresses the issues of inter-observer variability and the scarcity of local annotations through an approach that utilizes robust feature extractors (ResNet-50) and attention mechanisms for patch aggregation. The methodology integrated HSV color space preprocessing, embedding extraction, and training with slide-level labels on the public SICAPv2 dataset, employing patient-stratified cross-validation (GroupKFold) to ensure scientific rigor.\newline

The results achieved an \textbf{average F1-score of 0.88}, a \textbf{PR AUC of 0.917}, and a \textbf{Quadratic Weighted Kappa of 0.887}, demonstrating an "almost perfect" agreement with the medical gold standard. The system not only effectively classifies tumor tissue but also provides explainable outputs through attention maps that locate critical glandular structures (such as cribriform glands), aligning with Gleason scale criteria. A versioned and reproducible environment is provided to facilitate clinical validation and establish a solid baseline for computer-aided digital pathology.

\medskip
\textbf{Keywords:} Multiple Instance Learning (MIL); Whole Slide Images (WSI); Convolutional Neural Networks (CNN); Histopathology; Prostate Cancer; Explainable Artificial Intelligence (XAI).