% ====================== TÉRMINOS CLAVE ======================

\noindent
A continuación se presentan las definiciones de los términos centrales utilizados a lo largo de este trabajo, con el fin de establecer una base conceptual precisa y homogénea que facilite la comprensión de los capítulos posteriores.

\begin{enumerate}

    \item \textbf{Bag e instancia:} En el paradigma de \textit{Aprendizaje de Instancias Múltiples} (Multiple Instance Learning, MIL), un \textit{bag} corresponde a un conjunto de instancias individuales que no cuentan con etiquetas propias, pero que están asociadas colectivamente a una etiqueta de nivel superior \cite{carbonneau2018multiple}. Formalmente, el aprendizaje se realiza a partir de pares $(\mathcal{X}_i, y_i)$, donde $\mathcal{X}_i$ denota el conjunto de instancias y $y_i$ la etiqueta del bag. En el contexto de la histopatología digital, cada \textit{Whole Slide Image} (WSI) se modela como un bag, mientras que los parches extraídos de dicha imagen constituyen las instancias. Esta formulación resulta especialmente adecuada en escenarios clínicos reales, donde las anotaciones detalladas a nivel de parche suelen ser inexistentes o costosas de obtener.

    \item \textbf{Técnicas de agregación:} Corresponden a los mecanismos mediante los cuales la información proveniente de múltiples instancias dentro de un bag es combinada para producir una predicción global. Estas técnicas definen explícitamente la función de decisión del modelo MIL. Entre los enfoques más comunes se encuentran los esquemas clásicos de \emph{pooling}, como \emph{mean pooling} y \emph{max pooling}, que actúan como líneas base, así como métodos más avanzados basados en mecanismos de atención, los cuales aprenden a ponderar dinámicamente la contribución de cada instancia en función de su relevancia para la tarea de clasificación \cite{ilse2018attention, li2021dual, GADERMAYR2024102337}.

    \item \textbf{Supervisión débil:} Hace referencia a escenarios de aprendizaje en los que la información de etiquetado es incompleta, imprecisa o disponible únicamente a un nivel de granularidad superior, en contraste con la supervisión totalmente anotada a nivel de instancia \cite{zhou2018brief, campanella2019clinical, GADERMAYR2024102337}. En el caso de la clasificación de WSIs, las etiquetas suelen estar disponibles únicamente a nivel de lámina completa, lo que impone restricciones significativas al entrenamiento supervisado tradicional y motiva el uso de enfoques MIL como una solución viable y clínicamente realista.

    \item \textbf{Mecanismo de atención:} Es una estrategia de agregación que permite al modelo aprender pesos diferenciados para cada instancia dentro de un bag, reflejando su contribución relativa a la predicción final. En el contexto de MIL, los mecanismos de atención no solo mejoran la robustez del modelo frente a ruido y heterogeneidad intra-bag, sino que además proporcionan interpretabilidad intrínseca al evidenciar qué subconjuntos de instancias resultan más informativos para la decisión del modelo, lo cual es particularmente relevante en aplicaciones médicas \cite{ilse2018attention, lu2021data}.

    \item \textbf{Heterogeneidad intra-lámina:} Describe la variabilidad estructural, morfológica y semántica presente entre los parches que componen una misma WSI. Esta heterogeneidad puede deberse a la coexistencia de tejido sano, regiones tumorales de distinta agresividad, artefactos de preparación o zonas no informativas, y constituye uno de los principales desafíos en la clasificación automática de imágenes histopatológicas \cite{GADERMAYR2024102337}.

    \item \textbf{Embeddings o representaciones profundas:} Son vectores de características de dimensión reducida que codifican información semántica relevante extraída por redes neuronales convolucionales (CNNs) entrenadas como extractores de características. En este trabajo, cada parche es representado mediante un embedding en un espacio latente, lo que permite desacoplar la etapa de extracción de características de la etapa de agregación y clasificación basada en MIL \cite{litjens2017survey}.

    \item \textbf{ISUP y escala de Gleason:} La \emph{International Society of Urological Pathology} (ISUP) define un sistema de gradación del cáncer de próstata basado en la evaluación de patrones arquitecturales glandulares observados en tinciones hematoxilina-eosina (H\&E) \cite{epstein2016gleason, SILVARODRIGUEZ2020105637}. La escala de Gleason asigna puntuaciones que reflejan la agresividad tumoral, mientras que las categorías ISUP agrupan dichas puntuaciones en clases clínicamente significativas. En este trabajo, estas categorías son reformuladas en un esquema binario con el objetivo de abordar una tarea de clasificación alineada con decisiones diagnósticas prácticas y con la disponibilidad de datos.

    \item \textbf{Coeficiente de concordancia de Cohen (Kappa):} Es una métrica estadística utilizada para cuantificar el grado de acuerdo entre dos clasificadores discretos, corrigiendo por la probabilidad de coincidencia atribuible al azar. Esta medida se define como
    \[
        \kappa = \frac{p_o - p_e}{1 - p_e},
    \]
    donde \(p_o\) es la proporción de acuerdos observados y \(p_e\) es la proporción de acuerdos esperados por azar. El coeficiente Kappa toma valores entre \(-1\) y \(1\), donde valores cercanos a 1 indican un alto nivel de acuerdo, valores cercanos a $0$ sugieren acuerdo consistente con el azar, y valores negativos representan acuerdo menor al esperado por azar \cite{mchugh2012interrater}. Esta métrica se emplea frecuentemente en estudios clínicos y de clasificación para evaluar la concordancia entre predicciones automatizadas y anotaciones de referencia.

\end{enumerate}
