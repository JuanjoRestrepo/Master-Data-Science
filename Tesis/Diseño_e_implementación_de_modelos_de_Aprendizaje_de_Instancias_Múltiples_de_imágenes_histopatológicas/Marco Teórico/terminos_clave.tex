% ====================== TÉRMINOS CLAVE ======================

\noindent
A continuación se presentan definiciones de los términos centrales utilizados en este trabajo, con el fin de
establecer una base conceptual clara y homogénea para la comprensión de las secciones subsecuentes.


\begin{enumerate}
    \item \textbf{Bag e Instancia:} En el paradigma de \textit{Aprendizaje de Instancias Múltiples} (Multiple Instance
    Learning, MIL), una \textit{bag} representa un conjunto de instancias no etiquetadas individualmente, pero vinculadas a una etiqueta global de mayor nivel \cite{carbonneau2018multiple}. En el contexto de histopatología digital, cada \textit{Whole Slide Image} (WSI) se modela como un bag, mientras que los parches extraídos de dicha imagen corresponden a las instancias. Esta estructura permite que el aprendizaje se lleve a cabo aún cuando no existen anotaciones finas a nivel de parche, tal como es habitual en aplicaciones clínicas reales.

    \item \textbf{Técnicas de agregación:} Se refiere a los métodos para combinar información proveniente de múltiples instancias dentro de un mismo bag y producir una predicción global. Dentro de las técnicas más utilizadas se encuentran los esquemas clásicos de \emph{pooling} como el \emph{max pooling} y \emph{mean pooling}, así como arquitecturas más sofisticadas basadas en mecanismos de atención que asignan pesos variables a las instancias en función de su relevancia para la predicción final \cite{ilse2018attention, li2021dual, GADERMAYR2024102337}.

    \item \textbf{Supervisión débil:} Describe los escenarios de aprendizaje en los cuales la disponibilidad de etiquetas es parcial, inexacta o de bajo detalle, en contraposición a las anotaciones granulares de cada instancia individual \cite{zhou2018brief, campanella2019clinical, GADERMAYR2024102337}. En aplicaciones como la clasificación de WSIs, las etiquetas generalmente están disponibles solamente a nivel de lámina completa (slide-level), sin información precisa en los parches, lo que hace necesario el uso de paradigmas MIL para entrenar modelos bajo este tipo de restricción de supervisión.

    \item \textbf{Mecanismo de atención:} Es un método de agregación que aprende a asignar pesos diferenciados a cada instancia dentro de una bolsa en función de su contribución al resultado global. En MIL, los mecanismos de atención permiten descubrir qué subconjuntos de parches son más relevantes para la predicción, lo cual no solo mejora el rendimiento del modelo en presencia de ruido o heterogeneidad, sino que también habilita una forma de interpretabilidad post-hoc, al evidenciar regiones de mayor impacto diagnóstico \cite{ilse2018attention, lu2021data}.

    \item \textbf{Superposición y heterogeneidad:} Se refiere a la variación estructural y morfológica de los parches contenidos en un WSI, lo cual puede introducir ruido o complejidad adicional en la tarea de clasificación. Esta heterogeneidad surge debido a la presencia de regiones con tejido sano, artefactos o características no informativas, así como diferencias biológicas propias del tejido \cite{GADERMAYR2024102337}.

    \item \textbf{Embeddings o representaciones profundas:} Son vectores de características de dimensión reducida que resumen información semántica de alta dimensionalidad extraída por redes neuronales convolucionales (CNNs). Estos vectores permiten que cada parche sea representado como un punto en un espacio latente útil para tareas de clasificación o agregación \cite{litjens2017survey}.

    \item \textbf{ISUP y escala Gleason:} La \emph{International Society of Urological Pathology} (ISUP) define un sistema de gradación para el cáncer de próstata basado en la combinación de patrones arquitecturales glandulares observados en tinciones hematoxilina-eosina (H\&E) \cite{epstein2016gleason, SILVARODRIGUEZ2020105637}. La escala de Gleason asigna puntuaciones que reflejan la agresividad del tumor, y las categorías ISUP proporcionan agrupaciones clínicamente significativas. En este trabajo se emplea una reformulación binaria de estas categorías para facilitar decisiones diagnósticas relevantes.

    \item \textbf{Coeficiente de concordancia de Cohen (Kappa):} Métrica estadística que cuantifica el grado de acuerdo entre dos clasificaciones corrigiendo por la probabilidad de coincidencia aleatoria. En estudios clínicos con etiquetas discretas, el coeficiente Kappa es ampliamente utilizado para evaluar la concordancia real entre las predicciones del modelo y las anotaciones de referencia.

\end{enumerate}

