% ====================== REDES NEURONALES CONVOLUCIONALES ======================

Las \textbf{redes neuronales convolucionales (CNN)} constituyen una de las familias de modelos más utilizadas en visión por computador y en el campo del análisis de imágenes médicas, dado que permiten aprender automáticamente representaciones jerárquicas de características directamente de los datos de imagen \cite{CNNMedicalReview2025, DLDigitalPathologySurvey2020}.  
En particular, las CNN han revolucionado el análisis de imágenes histopatológicas por su capacidad de identificar patrones morfológicos complejos que son relevantes para tareas diagnósticas y subtipificación de tumores.\newline

En el contexto de la histopatología digital, las CNN se emplean principalmente como \textbf{extractores de características a nivel de parche} debido a que las imágenes de diapositivas completas (WSIs) son demasiado grandes para ser procesadas directamente por modelos de visión estándar. Convertir cada parche en un \textbf{vector de características (embedding)} permite capturar información local relevante, como texturas, estructuras celulares y patrones tisulares, que luego puede utilizarse como entrada para modelos de agregación posteriores \cite{HistopathologyCNNFeatures2017, DLDigitalPathologySurvey2020, PatchBasedWSICNN2016}.\newline

\newpage
A continuación, se presenta la Figura \ref{fig:cnn_arch} que ilustra la arquitectura típica de una CNN.
\begin{figure}[h!]
    \centering
    \includegraphics[width=0.6\textwidth]{images/CNN_Typical_Arch.png}
    \caption{Arquitectura típica de una red neuronal convolucional, con capas de convolución, activación y \textit{pooling} que permiten extraer características jerárquicas \cite{Géron_2022}.}
    \label{fig:cnn_arch}
\end{figure}

Una estrategia comúnmente adoptada en aplicaciones de histopatología digital es el uso de \textbf{transfer learning}, donde arquitecturas de CNN pre-entrenadas en conjuntos de datos extensos como ImageNet son adaptadas para tareas específicas de clasificación o extracción de características sobre parches de tejido \cite{HistopathologyCNNFeatures2017, DLDigitalPathologySurvey2020}.  
Modelos como ResNet, VGG y EfficientNet han demostrado ser eficaces como extractores de características en dominios médicos, reduciendo el tiempo de entrenamiento y mejorando la generalización cuando la cantidad de datos anotados es limitada. \newline

Durante el proceso de entrenamiento de los modelos, se emplean técnicas de \textbf{aumentación de datos},  como rotaciones, volteos y variaciones de color, para expandir de forma virtual el conjunto de entrenamiento y mitigar el riesgo de sobreajuste \cite{CNNMedicalReview2025}. Además, la normalización de tinción y ajustes de color suelen aplicarse como pasos adicionales de preprocesamiento en histopatología digital para reducir la variación entre diferentes instituciones o dispositivos de escaneo. \newline

En el marco de \textbf{Aprendizaje de Instancias Múltiples (MIL)}, los embeddings extraídos por la CNN de cada parche son posteriormente combinados mediante mecanismos de agregación, como el \textit{pooling} o mecanismos de atención, para generar una predicción a nivel de WSI sin requerir anotaciones localizadas exhaustivas \cite{DLDigitalPathologySurvey2020}.
