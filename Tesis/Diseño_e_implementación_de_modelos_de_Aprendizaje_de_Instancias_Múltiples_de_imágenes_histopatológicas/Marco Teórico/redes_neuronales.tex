Las redes neuronales convolucionales (CNN) se han establecido como el paradigma dominante en la visión por computadora debido a su capacidad intrínseca para aprender jerarquías de características espaciales y patrones visuales complejos \cite{Géron_2022}. En el contexto de la histopatología digital, las CNN se emplean fundamentalmente como extractores de características a nivel de parche, lo cual es vital para procesar las imágenes de diapositivas completas (WSI) dada su gran escala.\newline

Cada parche es procesado individualmente por la CNN para generar una representación vectorial (embedding) que encapsula la información morfológica relevante, como las texturas, estructuras glandulares y patrones tisulares. Estas representaciones permiten reducir la subjetividad inherente a la interpretación visual humana y constituyen la entrada fundamental para los modelos MIL \cite{lu2021data} \cite{HEZI2025108513}.\newline

A continuación, se presenta la figura \ref{fig:cnn_arch} que ilustra la arquitectura típica de una CNN.\newpage

\begin{figure}[h]
    \centering
    \includegraphics[scale=0.3]{images/CNN_Typical_Arch.png}
    \caption{Arquitectura típica de una red neuronal convolucional \cite{Géron_2022}}
    \label{fig:cnn_arch}
\end{figure}

Una estrategia comúnmente adoptada es el uso de transfer learning con arquitecturas pre-entrenadas en grandes bases de datos como ImageNet. Modelos como ResNet o EfficientNet son adaptados para la tarea de clasificación histopatológica, lo cual acelera la convergencia del entrenamiento y mejora la generalización en conjuntos de datos de tamaño moderado \cite{lu2021data} \cite{he2016deep} \cite{tan2019efficientnet}. Para estabilizar el proceso de aprendizaje y mitigar el sobreajuste, es crucial incorporar técnicas de regularización y estrategias de aumentación de datos diseñadas específicamente para el dominio histopatológico.\newline

En el marco del Aprendizaje de Instancias Múltiples (MIL), las características vectoriales extraídas por la CNN de cada parche son posteriormente agregadas por módulos especializados, como los mecanismos de atención o el pooling jerárquico. Esto permite generar una predicción a nivel de la lámina completa, eliminando la necesidad de un etiquetado exhaustivo de cada parche y consolidando el pipeline de análisis de WSI.