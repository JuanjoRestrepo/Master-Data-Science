La investigación en aprendizaje débilmente supervisado sobre Whole Slide Images (WSI) ha avanzado rápidamente en los últimos años, motivada por la necesidad de reducir la carga de anotación experta y por el potencial clínico de modelos reproducibles y explicables. Varios trabajos han propuesto arquitecturas y estrategias que permiten agregar evidencia local (parches) en predicciones a nivel de muestra (bag), y han explorado alternativas para incorporar contexto espacial, medidas de incertidumbre y visualizaciones de interpretabilidad. \newline

Para contextualizar la presente investigación, a continuación se revisan los antecedentes más relevantes que abordan estas problemáticas. Dichos estudios justifican las decisiones metodológicas de este proyecto al ilustrar las principales tendencias y desafíos en el campo, sirviendo como base teórica para el diseño de la solución propuesta:

\subsection{CAMIL: channel attention-based multiple instance learning 
for whole slide image classification}
En este trabajo, los autores proponen una arquitectura denominada FA-MILNet (Feature Aggregation Multiple Instance Learning Network), orientada a abordar los desafíos del aprendizaje débilmente supervisado en la clasificación de Whole Slide Images (WSI). La propuesta surge como respuesta a las limitaciones de los enfoques MIL tradicionales, los cuales suelen perder información discriminativa al emplear estrategias de agregación simples como el max pooling y no modelan adecuadamente la heterogeneidad intralaminar.\newline

FA-MILNet introduce un esquema de agregación jerárquica asistida por atención, integrando información local y global mediante tres componentes principales: extracción de características basada en CNN, un módulo de atención para ponderar la relevancia de los parches y una rama dual de agregación. Esta arquitectura permite mejorar simultáneamente el desempeño predictivo y la interpretabilidad del modelo \cite{Mao2025_CAMIL}.\newline

Evaluado en conjuntos de datos de cáncer de colon, mama y ganglios linfáticos (Camelyon16, BACH y CRC), el modelo supera enfoques como ABMIL, CLAM y DSMIL en métricas como AUC y F1-score. Si bien el estudio no se enfoca en cáncer de próstata, su diseño modular y su capacidad para operar sin anotaciones patch-level lo convierten en un antecedente altamente pertinente para este proyecto, al validar el uso de mecanismos de atención como estrategia central para la agregación de información en escenarios de supervisión débil \cite{Mao2025_CAMIL}.\newline


% --------------------------------------------------------------------------

\subsection{A Probabilistic MIL Model with Spatial Regularization for Weakly Supervised Histopathology Image Analysis}
Este trabajo presenta un modelo de Aprendizaje de Instancias Múltiples de carácter probabilístico, basado en procesos gaussianos e integrado con un término de regularización espacial inspirado en el modelo de Ising. La propuesta parte de la hipótesis de que los parches histológicamente cercanos presentan una alta correlación diagnóstica, por lo que modelar explícitamente estas dependencias espaciales puede mejorar la robustez del aprendizaje bajo supervisión débil \cite{MoralesAlvarez2024}.\newline

A diferencia de los modelos MIL determinísticos, esta aproximación permite estimar distribuciones de probabilidad sobre las etiquetas de las instancias, proporcionando una cuantificación explícita de la incertidumbre asociada a las predicciones. Esta característica resulta particularmente relevante en contextos clínicos, donde la identificación de casos ambiguos y la evaluación de la confianza del modelo son aspectos críticos para su adopción práctica.

% --------------------------------------------------------------------------

\subsection{Graph-based Multiple Instance Learning for Whole Slide Image Classification}
En este estudio se propone una variante de MIL basada en grafos, donde cada parche de una WSI se modela como un nodo y las relaciones espaciales entre parches se representan mediante aristas en un grafo no dirigido. Sobre esta estructura se emplea una Graph Neural Network (GNN), permitiendo la propagación de información contextual entre regiones del tejido y facilitando la captura de patrones arquitectónicos globales \cite{Fourkioti2023}. \newline

Este enfoque ofrece ventajas claras en términos de interpretabilidad estructural y modelado explícito del contexto espacial. Sin embargo, introduce una complejidad computacional significativa asociada tanto a la construcción del grafo como al entrenamiento del modelo, además de requerir decisiones de diseño adicionales sobre la conectividad y los criterios de vecindad.\newline

Dichas consideraciones exceden el alcance experimental del presente proyecto, lo que motivó la selección de arquitecturas más ligeras basadas en mecanismos de atención directa, que permiten un análisis interpretativo más sencillo y una evaluación reproducible bajo restricciones computacionales realistas \cite{Fourkioti2023}.



% --------------------------------------------------------------------------

\subsection{Deep Recurrent Attention MIL for Histopathological Image Analysis}
El enfoque planteado en este artículo se distingue por combinar mecanismos de atención con redes neuronales recurrentes (RNN) dentro del marco de MIL. En lugar de tratar los parches como elementos independientes, los autores introducen una secuencia ordenada basada en la posición espacial de cada uno, la cual es procesada por una RNN con atención. Este diseño permite capturar no solo las características locales, sino también las relaciones de largo alcance y las dependencias contextuales más profundas a lo largo de toda la WSI \cite{Qu_Yang_Huang_Guo_Luo_Zhang_Wang_2024}. \newline

Gracias a esta arquitectura secuencial, el modelo demuestra un mejor rendimiento en la identificación de estructuras glandulares complejas, un desafío particular en escenarios con gran variabilidad morfológica entre pacientes. Además, el mecanismo de atención adaptativa complementa este diseño, facilitando interpretaciones más precisas del modelo al resaltar de forma dinámica las regiones más relevantes. La principal lección de este trabajo es que la secuencia y el orden de los parches, y no solo su presencia, pueden aportar información crucial. A diferencia de los enfoques basados en grafos que modelan la vecindad de forma explícita, este método explora una representación secuencial, lo que invita a considerar cómo la estructura de los datos de entrada puede ser modificada para extraer de manera más efectiva la información contextual \cite{Qu_Yang_Huang_Guo_Luo_Zhang_Wang_2024}.

% --------------------------------------------------------------------------

\subsection{Hierarchical Pooling in MIL for Histopathological Subtype Classification}
Este trabajo propone un esquema de agregación jerárquica dentro del paradigma MIL, en el cual los parches primero se agrupan en regiones intermedias según su proximidad o similitud morfológica. Posteriormente, se aplica un segundo nivel de pooling que combina la información de estas regiones para obtener la predicción final del bag. Esta estructura multinivel permite capturar patrones de organización tisular más amplios, que son clave en el diagnóstico de ciertos subtipos de cáncer. La metodología incorpora tanto técnicas de agrupamiento automático como atención regional, y ha mostrado un desempeño superior en conjuntos de datos con alta heterogeneidad intratumoral, al mantener el balance entre granularidad local y contexto estructural \cite{waqas2024survey}. \newline

Este antecedente es relevante porque, al igual que FA-MILNet, enfatiza la importancia de una agregación multinivel. Sin embargo, se distingue por su énfasis en la creación de ''grupos'' lógicos de parches antes de la agregación final, lo que es una estrategia diferente a la simple ponderación por atención. Por otro lado. la idea de agrupar parches basándose en la morfología o la proximidad podría ser una técnica poderosa a considerar para proyectos que buscan capturar la complejidad de la heterogeneidad tumoral \cite{waqas2024survey}.


\subsection{Síntesis crítica y posicionamiento del presente trabajo}
La revisión del estado del arte evidencia que el Aprendizaje de Instancias Múltiples (MIL) aplicado a Whole Slide Images ha experimentado una evolución sustancial, incorporando mecanismos de atención, modelado espacial explícito y esquemas de agregación jerárquica. Estas aproximaciones han demostrado mejoras significativas tanto en desempeño predictivo como en capacidad interpretativa, particularmente en escenarios caracterizados por supervisión débil y elevada heterogeneidad morfológica.\newline

No obstante, muchos de estos métodos de última generación, tales como CLAM o arquitecturas MIL basadas en Transformers, dependen de componentes adicionales como pseudo-etiquetado, clustering previo, modelado explícito de grafos o mecanismos de autoatención de alta complejidad. Si bien estas estrategias pueden mejorar el rendimiento en determinados contextos, también introducen un incremento considerable en el costo computacional, la sensibilidad a hiperparámetros y la dificultad de interpretación directa, factores que limitan su aplicabilidad práctica y su evaluación reproducible en entornos clínicos reales.\newline

Bajo esta premisa, el presente trabajo se posiciona deliberadamente en un punto de equilibrio entre capacidad predictiva, interpretabilidad y viabilidad experimental. En particular, se adopta un marco de MIL basado en mecanismos de atención explícitos (ABMIL y SmABMIL), los cuales permiten ponderar la contribución relativa de cada parche a la predicción global sin requerir anotaciones a nivel de instancia ni imponer estructuras espaciales adicionales. Esta elección metodológica se alinea con las restricciones reales de disponibilidad de datos en histopatología digital y facilita un análisis interpretativo claro y directamente asociado al proceso de decisión del modelo.\newline

Asimismo, este enfoque establece una base sólida y extensible para trabajos futuros, en los cuales podrían incorporarse modelos jerárquicos, probabilísticos o basados en grafos una vez se disponga de mayores recursos computacionales o anotaciones complementarias, preservando siempre la coherencia clínica y la trazabilidad interpretativa del sistema propuesto.

