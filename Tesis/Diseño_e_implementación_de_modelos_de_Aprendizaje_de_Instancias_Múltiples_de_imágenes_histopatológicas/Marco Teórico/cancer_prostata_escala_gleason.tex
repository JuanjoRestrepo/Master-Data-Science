% ================= CÁNCER DE PRÓSTATA Y ESCALA DE GLEASON =================

El cáncer de próstata es una de las neoplasias más frecuentes y una de las principales causas de morbilidad y mortalidad en la población masculina a nivel mundial \cite{SungH_2021_Global, MinSalud_2025}. La detección temprana y precisa de esta enfermedad es fundamental para orientar decisiones terapéuticas y estrategias de manejo individualizado, debido a la amplia variabilidad biológica de la enfermedad y su impacto en la supervivencia a largo plazo.\newline

El diagnóstico clínico convencional se basa en el análisis histopatológico de biopsias teñidas con hematoxilina y eosina (H\&E), donde el patólogo examina patrones arquitecturales del tejido prostático bajo el microscopio.  
Este examen produce la asignación de un \textbf{puntaje de Gleason}, un sistema de gradación que clasifica la agresividad tumoral con base en la arquitectura glandular observada en las muestras \cite{Brett_2016, Ozkan2016}. \newline

La escala de Gleason combina dos patrones arquitecturales, primario y secundario, para obtener un puntaje sumado que refleja la agresividad biológica del tumor.  
\begin{itemize}
    \item Puntajes bajos (por ejemplo, Gleason 6) se asocian con tumores bien diferenciados y menor agresividad.  
    \item Puntajes altos (Gleason 8–10) indican tumores más indiferenciados y de mayor agresividad clínica.  
\end{itemize}

Esta información se resume y agrupa en categorías clínicamente relevantes definidas por la \textbf{International Society of Urological Pathology (ISUP)} \cite{epstein2016gleason, SILVARODRIGUEZ2020105637}.\newline

Ahora bien, un patrón histológico particularmente desafiante es el \textbf{cribriforme}, caracterizado por estructuras glandulares complejas que se asocian con peor pronóstico y mayor riesgo de recurrencia. Numerosos estudios han documentado que la presencia de patrones cribiformes se correlaciona con desenlaces clínicos adversos, lo que subraya la importancia de su identificación precisa \cite{SILVARODRIGUEZ2020105637}. La Figura \ref{fig:cribriform_patterns} muestra ejemplos de este patrón histológico, evidenciando la complejidad de su arquitectura glandular y la dificultad diagnóstica que representa.

\begin{figure}[h!]
    \centering
    \includegraphics[width=0.4\textwidth]{images/gleason_patterns/Four-basic-architectural-patterns-of-Gleason-pattern-4.jpg}
    \caption{Ejemplos representativos de patrones cribiformes (Gleason 4) en tejido prostático \cite{SILVARODRIGUEZ2020105637}.}
    \label{fig:cribriform_patterns}
\end{figure}

% ### **Justificación clínica de la binarización ISUP**
A pesar de la importancia clínica de la gradación de Gleason, su asignación presenta variabilidad \textbf{interobservador} significativa, incluso entre patólogos experimentados \cite{Ozkan2016, Brett_2016, GleasonInterobserverPubMed2023}. Dicha variabilidad es especialmente notoria en la distinción entre \textbf{Gleason 3 y Gleason 4} y en la identificación de patrones con riesgo clínico intermedio o alto, lo que genera retos en la reproducibilidad y consistencia diagnóstica.\newline

Para integrar esta complejidad diagnóstica con métodos computacionales basados en aprendizaje automático, es frecuente reformular el problema en términos clínicos significativos. Una aproximación clínicamente respaldada consiste en \textbf{binarizar las categorías ISUP} en dos grupos de interés:

\begin{itemize}
    \item \textbf{Bajo grado}: tumores con menor agresividad (ISUP grados 1–2).
    \item \textbf{Alto grado}: tumores con mayor probabilidad de progresión clínica adversa (ISUP grados ≥3).
\end{itemize}

Esta reformulación reduce la ambigüedad diagnóstica en casos limítrofes y se alinea con decisiones clínicas reales, tales como la elección entre tratamientos activos frente a intervenciones más agresivas. Diversos trabajos en patología computacional han adoptado esquemas similares cuando el objetivo es la \textbf{clasificación binaria de riesgo clínico} en cáncer de próstata \cite{campanella2019clinical, MoralesAlvarez2024}. \newline


% ### **Necesidad de enfoques computacionales**
La digitalización de imágenes histopatológicas (WSIs) genera grandes volúmenes de información visual de alta resolución \cite{litjens2017survey}. Procesar y analizar estas imágenes exhaustivamente supera las capacidades humanas en términos de rapidez y consistencia, especialmente frente a:

\begin{itemize}
    \item Variabilidad morfológica intrínseca del tejido,
    \item Artefactos de preparación y tinción,
    \item La elevada cantidad de parches contenidos en una sola WSI.
\end{itemize}

Estas limitaciones han motivado la adopción de técnicas computacionales avanzadas, particularmente en aprendizaje profundo, para apoyar la evaluación histopatológica y mejorar la reproducibilidad diagnóstica. Las arquitecturas de \textbf{Multiple Instance Learning} (MIL), junto con mecanismos de atención, se presentan como soluciones naturales para integrar supervisión débil con representación de patrones discriminativos en WSIs (ver Sección \ref{sec:aprendizaje_MIL}).\newline

En consecuencia, el desarrollo de modelos computacionales capaces de imitar y fortalecer el juicio clínico en la gradación de Gleason constituye un área de investigación activa y de alto impacto clínico. La variabilidad interobservador, la complejidad morfológica de ciertos patrones histológicos y la necesidad de evaluaciones reproducibles han impulsado el uso de enfoques basados en aprendizaje profundo para apoyar el diagnóstico histopatológico.


%Esta complejidad en la gradación del cáncer de próstata, junto con la variabilidad interobservador y la necesidad de una evaluación rápida y reproducible, motiva la adopción de técnicas de histopatología digital.  
%El análisis de imágenes de lámina completa (Whole Slide Images, WSI) permite capturar información morfológica a alta resolución, integrando patrones microscópicos críticos como los observados en los puntajes de Gleason.  
%En la sección siguiente se abordarán los conceptos, características y desafíos de la histopatología digital, estableciendo el marco para la aplicación de modelos computacionales de aprendizaje de instancias múltiples en la clasificación de tejido prostático.




\begin{comment}
    
\begin{figure}[h]
    \centering
    \includegraphics[width=0.4\textwidth]{images/Four-basic-architectural-patterns-of-Gleason-pattern-4-A-Cribriform-glands-B-Fused_W640.jpg}
    % \includegraphics[width=0.30\textwidth]{images/Example-of-a-cribriform-pattern_W640.jpg}
    % \includegraphics[width=0.35\textwidth]{images/Gleason grade 4 patterns and intraductal carcinoma.jpg}
    % \includegraphics[width=0.35\textwidth]{images/cancers-14-03041-g001.png}
    \caption{Ejemplos representativos de patrones cribiformes (Gleason 4) en tejido prostático \cite{SILVARODRIGUEZ2020105637}.}
    \label{fig:cribriform_patterns}
\end{figure}

\end{comment}


