%%
%% This is file `monografia.tex'
%%

%%%%%%%%%%%%%%%%
%% Plantilla para la presentaci\'on de monograf\'ias de trabajo de grado
%% Carrera de Ingenier\'ia Electr\'onica
%% Pontificia Universidad Javeriana - Cali
%% Mayo 2020
%%%%%%%%%%%%%%%%

%\documentclass[11pt,twoside]{thesis-tg-ie-pujc}
\documentclass{thesis-tg-ie-pujc} % Se utiliza la nueva clase
\usepackage[utf8]{inputenc}
\usepackage[T1]{fontenc}
\usepackage{enumerate}
\usepackage{longtable}
\usepackage{xurl}
\usepackage[breaklinks]{hyperref}
\usepackage{makeidx}
\usepackage{multicol}
\usepackage{graphics}
\usepackage{pdflscape} % provides the landscape environment
\usepackage{ragged2e} % provides \RaggedLeft
\usepackage[intoc, spanish]{nomencl}
\usepackage[spanish]{babel}
% Nota: la clase/paquetes ya proporcionan \captionof; se eliminó \usepackage{capt-of} para evitar conflicto

%%%%%%%%%%%%%%%%%%
%%
%% This is file `author_data.tex'
%%

%%%%%%%%%%%%%%%%
%% Datos a modificar por los autores del proyecto de trabajo de grado
%% Carrera de Ingeniería Electrónica
%% Pontificia Universidad Javeriana – Cali
%% Julio 2025
%%%%%%%%%%%%%%%%

%%%%%%%%%%%%%%%%
% TÍTULO DEL PROYECTO
%%%%%%%%%%%%%%%%
\titulo
{Diseño e implementación de modelos de Aprendizaje de Instancias Múltiples para la clasificación débilmente supervisada de imágenes histopatológicas de cáncer de próstata}


%%%%%%%
% AUTOR
%%%%%%%
\autorA{8939280}{Juan José Restrepo Rosero}
\autorB{8992284}{María Valentina Belalcázar Perdomo}

%%%%%%%
% FECHA DE ENTREGA
%%%%%%%
\fecha{\today}

%%%%%%%%%%%%%%%%%
% METADATA PDF
% TÍTULO DEL TRABAJO DE GRADO
% AUTOR(ES)
% KEYWORDS
%%%%%%%%%%%%%%%%%
\hypersetup
{
  pdftitle={Proyecto X},
  pdfauthor={Juan José Restrepo Rosero, María Valentina Belalcázar Perdomo},
  pdfkeywords={Creación de Valor Compartido, Sostenibilidad, Pymes, Data Engineering, Multi-omics}
}

%%%%%%%%%%%%%%%%%
% DIRECTOR DEL PROYECTO
%%%%%%%%%%%%%%%%%
\directorproyecto{Dr.}{Julian Gil González}

%%%%%%%%%%%%%%%%%
% DECANO FACULTAD DE INGENIERÍA Y CIENCIAS
% DIRECTOR DE CARRERA
%%%%%%%%%%%%%%%%%
% \decanoFIC{Dr.}{Hernán Camilo Rocha Niño}
\directorcarrera{Dr.}{Diego Luis Linares Ospina}

%%%%%%%
% VERSIÓN
%%%%%%%
\version{DEFINITIVA}

%%%%%%%
% JURADOS
%%%%%%%
\juradoA{Dr.}{Nombre Apellido} % Jurado 1
\juradoB{Dr.}{Nombre Apellido} % Jurado 2

\include{thesis-tg-ie-pujc}
\makeindex

%%%%%%%%%%%%%%%%%%
% INICIO DEL DOCUMENTO
%%%%%%%%%%%%%%%%%%
\begin{document}

\maketitle
\makecoverletter

%%%%%%%%%%%%%%%%%% DEDICATORIA %%%%%%%%%%%%%%%%%%
\chapter*{Dedicatoria}
El presente trabajo de grado se lo dedicamos a Dios, familiares y a amigos cercanos que nos han acompañado en esta experiencia de formación como Ingenieros Electrónicos.

%%%%%%%%%%%% AGRADECIMIENTOS %%%%%%%%%%%%%%%%%
\chapter*{Agradecimientos }
% Agradecemos especialmente a nuestros directores de grado, que han estado muy pendientes del progreso, nos han servido de soporte y guía para la realización de este proyecto. Agradecemos especialmente su constante apoyo durante los casi 12 meses de desarrollo e implementación del proyecto.\\

% Además, agradecemos a todos los docentes de la carrera, que a partir de sus enseñanzas, nos permitieron lograr alcanzar el título como Magíster en Ciencia de Datos. Agradecemos a nuestras familias quiénes nos han apoyado en este proceso con sus consejos, además, han permitido que podamos seguir un buen camino para su realización y han estado constantemente pendiente del desarrollo de nuestro trabajo de grado.




%%%%%%%%%%%%%%%%%%
% ABSTRACT
%%%%%%%%%%%%%%%%%%
\chapter*{Resumen}
Las pequeñas y medianas empresas (Pymes) constituyen la columna vertebral de la economía colombiana, desempeñando un papel crucial en la generación de empleo y en el dinamismo del tejido empresarial. Sin embargo, estas organizaciones enfrentan una serie de desafíos importantes para integrar prácticas sostenibles en sus operaciones. Entre ellos se destacan la limitada capacidad financiera, el difícil acceso a tecnologías limpias, la escasa formación en gestión ambiental y la falta de incentivos gubernamentales. Estos factores, combinados con la presión constante por mantenerse competitivas en mercados cada vez más exigentes, hacen que muchas Pymes prioricen sus estrategias operativas y de crecimiento económico, dejando de lado consideraciones ambientales que podrían contribuir significativamente a la creación de valor compartido (CVC).\\

En respuesta a este panorama, el proyecto se propone desarrollar un modelo multivariante que permita predecir la creación de valor compartido y su relación con la sostenibilidad ambiental en las Pymes colombianas. Inspirándose en la integración de variables clave que reflejen tanto la situación interna de cada empresa (como la capacidad operativa, disponibilidad de recursos y nivel de conocimiento técnico) como factores externos del entorno (como las dinámicas del mercado, políticas gubernamentales y tendencias en sostenibilidad), este modelo se plantea como una herramienta predictiva y práctica para la toma de decisiones estratégicas. La idea es que, mediante el análisis estadístico y el uso de técnicas avanzadas de modelado, se pueda identificar qué variables tienen mayor influencia en el éxito de la implementación de prácticas sostenibles y, de esta forma, generar un marco de referencia que oriente a las Pymes hacia una transformación empresarial integral.\\

El enfoque propuesto no solo busca mejorar la competitividad de las Pymes, sino también impulsar una reorientación de sus procesos hacia una economía más resiliente y responsable. Con este modelo, se espera que las empresas no solo puedan optimizar su rendimiento económico, sino que también generen beneficios sociales y ambientales, fortaleciendo el concepto de creación de valor compartido. Así, el proyecto contribuye a una mayor comprensión de los retos y oportunidades en la integración de prácticas sostenibles, ofreciendo una solución innovadora que, además de facilitar la toma de decisiones, promueve la adopción de estrategias que benefician tanto a la empresa como a la sociedad en general.\\

En síntesis, este estudio se posiciona como una respuesta integral a las limitaciones actuales en la implementación de medidas de sostenibilidad en las Pymes colombianas, al combinar el análisis predictivo con la evaluación de variables contextuales. Se espera que la aplicación de este modelo multivariante no solo genere mejoras en la eficiencia operativa y competitividad de las empresas, sino que también contribuya de manera decisiva al desarrollo sostenible del país, al facilitar un alineamiento entre los objetivos económicos y las necesidades ambientales y sociales.\\

\textbf{Palabras Claves:} Creación de Valor Compartido, Modelos Multivariantes, Pymes, Sostenibilidad Ambiental.

% %Base de datos de grafos, Celdas de manufactura, Neo4j, Cypher, Sistemas de ejecución de manufactura.

\chapter*{Abstract}
This project presents the design and implementation of a reproducible pipeline based on Multiple Instance Learning (MIL) for the weakly supervised classification of prostate cancer Whole Slide Images (WSI). The study addresses the issues of inter-observer variability and the scarcity of local annotations through an approach that utilizes robust feature extractors (ResNet-50) and attention mechanisms for patch aggregation. The methodology integrated HSV color space preprocessing, embedding extraction, and training with slide-level labels on the public SICAPv2 dataset, employing patient-stratified cross-validation (GroupKFold) to ensure scientific rigor.\newline

The results achieved an \textbf{average F1-score of 0.88}, a \textbf{PR AUC of 0.917}, and a \textbf{Quadratic Weighted Kappa of 0.887}, demonstrating an "almost perfect" agreement with the medical gold standard. The system not only effectively classifies tumor tissue but also provides explainable outputs through attention maps that locate critical glandular structures (such as cribriform glands), aligning with Gleason scale criteria. A versioned and reproducible environment is provided to facilitate clinical validation and establish a solid baseline for computer-aided digital pathology.

\medskip
\textbf{Keywords:} Multiple Instance Learning (MIL); Whole Slide Images (WSI); Convolutional Neural Networks (CNN); Histopathology; Prostate Cancer; Explainable Artificial Intelligence (XAI).

%%%%%%%%%%%%%%%%%
%%%% TABLE OF CONTENTS
%%%% AND PREAMBLE
%%%%%%%%%%%%%%%%%

\tableofcontents
\listoffigures

%%%%%%%%%%%%%%%%%
%%%% MAIN MATTER
%%%%%%%%%%%%%%%%%

\dominitoc
\mainmatter

%%%%%%% INTRODUCCION %%%%%%%%%%%%%%%%%%b 
\chapter{Introducción}
% El Semillero de Investigación S-iÓMICAS, adscrito al Instituto iÓMICAS de la Pontificia Universidad Javeriana Cali, se presenta como un espacio de colaboración interdisciplinaria. Su misión primordial es abordar y resolver problemáticas complejas en áreas críticas como la salud humana y vegetal, la nutrición y la biotecnología, con un énfasis particular en las ciencias ómicas. A través de este espacio académico se busca fortalecer las competencias investigativas de estudiantes y docentes, promoviendo tanto la generación de conocimiento novedoso como el desarrollo de aplicaciones tecnológicas de alto impacto. \cite{iomicas2025}.\\

% En un escenario caracterizado por el crecimiento exponencial de datos generados por equipos de laboratorio y sensores especializados (como los utilizados en microscopía, espectroscopía y secuenciación), se hace imprescindible la adopción de sistemas de gestión del conocimiento capaces de procesar y armonizar información heterogénea, volumétrica y dispersa. Dichos sistemas deben integrar tecnologías emergentes como el Internet de las Cosas (IoT) y la inteligencia artificial (IA), con el fin de convertir datos brutos en conocimiento estructurado, contextualizado y procesable para la toma de decisiones estratégicas \cite{kajiyama2019km_iot}\cite{case_iot}.\\

% Para abordar esta necesidad de manera sistemática, se necesita adoptar el marco metodológico CDIO (\emph{Concebir, Diseñar, Implementar y Operar}), ampliamente reconocido en marcos educativos de ingeniería \cite{cdio2022}. Este enfoque enfatiza la importancia de integrar la gestión de datos y el conocimiento como elementos fundamentales en el diseño, implementación y operación de sistemas complejos.\\

% El presente documento tiene como objetivo definir e implementar una metodología integrada para el Semillero S‑iÓMICAS que permita:
% \begin{enumerate}
%     \item Capturar de manera sistemática el conocimiento tácito y operativo generado en las actividades de investigación.
%     \item Estandarizar datos y documentos mediante un sistema estructurado, enriquecido con metadatos según estándares FAIR.
%     \item Implementar modelos de análisis basados en IA que identifiquen patrones, tendencias y relaciones relevantes en los datos ómicos.
%     \item Desplegar, evaluar y retroalimentar continuamente el sistema en el entorno real del semillero.
% \end{enumerate}

% El documento se estructura en cuatro capítulos principales, cada uno alineado con una fase del modelo CDIO:
% \begin{itemize}
%     \item \textbf{Concebir}: Levantamiento de información, diagnóstico inicial e identificación de necesidades del entorno.
%     \item \textbf{Diseñar}: Definición del modelo de datos, estructura de metadatos y arquitectura tecnológica.
%     \item \textbf{Implementar}: Desarrollo de pipelines de procesamiento y estructuración de datos.
%     \item \textbf{Operar}: Puesta en marcha, monitoreo, validación de resultados y mejora continua del sistema.
% \end{itemize}

% Con esta aproximación, se busca no solo preservar y visibilizar el conocimiento científico generado por el semillero, sino también maximizar su impacto a través del uso estratégico de tecnologías emergentes y principios de ciencia abierta.


%%%%%%%%%%%%%%%%%% PLANTEAMIENTO DEL PROBLEMA %%%%%%%%%%%%%%%%%%%%
\chapter{Descripción del Problema}
El cáncer de próstata constituye una de las principales causas de morbilidad y mortalidad en la población masculina a nivel mundial. Según estimaciones recientes de la Organización Mundial de la Salud (OMS) y del Global Cancer Observatory (GCO), se reportan anualmente más de 1.4 millones de casos nuevos, con una carga especialmente elevada en países de ingresos medios y bajos debido a barreras en el acceso al diagnóstico temprano y al tratamiento oportuno \cite{SungH_2021}. En el contexto colombiano, esta enfermedad figura entre las primeras causas de muerte por cáncer en hombres, lo que subraya la necesidad de fortalecer estrategias de detección precoz y estratificación de riesgo que apoyen la toma de decisiones clínicas \cite{MinSalud_2025}.\newline

El diagnóstico histopatológico mediante la escala de Gleason constituye el pilar de la estratificación pronóstica; sin embargo, su aplicación está sujeta a una variabilidad interobservador significativa. Esta subjetividad es especialmente crítica en la discriminación entre patrones limítrofes (por ejemplo, Gleason 3 frente a Gleason 4) y en la identificación de subtipos histológicos asociados a peor pronóstico, como el patrón \emph{cribriforme}. Dicho patrón se caracteriza por formaciones glandulares con arquitectura tipo criba y se asocia a mayor agresividad tumoral y riesgo de recurrencia, por lo que su correcta detección tiene implicaciones terapéuticas relevantes \cite{SILVARODRIGUEZ2020105637} \cite{Brett_2016}.\newline

La digitalización de muestras histopatológicas y la generación de \textit{Whole Slide Images} (WSI) han habilitado el análisis computacional del tejido completo, pero introducen retos técnicos sustanciales. Entre ellos se incluyen el tamaño gigapíxel de las imágenes, la heterogeneidad morfológica intratumoral, las variaciones de tinción entre centros y la presencia de artefactos derivados de la preparación y el escaneo de las muestras \cite{Tellez_2018}. Adicionalmente, la disponibilidad de conjuntos de datos con anotaciones detalladas a nivel de parche (\textit{patch-level}) es limitada, debido al alto costo y tiempo requeridos para su generación por expertos.\newline

Es importante aclarar la diferencia entre los niveles de anotación que condicionan la metodología de aprendizaje. Una etiqueta a nivel de lámina/muestra (\textit{slide-level}) asigna una única etiqueta/clase a toda la WSI, mientras que una etiqueta a nivel de instancia (\textit{patch-level}) identifica la clase de cada parche individual. La ausencia generalizada de anotaciones \textit{patch-level} ha motivado el uso de enfoques de aprendizaje débil, particularmente el paradigma de \textit{Multiple Instance Learning}, que permite entrenar modelos utilizando únicamente etiquetas globales a nivel de WSI.\newline

Desde la perspectiva de la ciencia de datos, la combinación de imágenes de gran escala, supervisión débil y heterogeneidad interinstitucional genera un conjunto de desafíos metodológicos bien definidos. Estos incluyen la necesidad de estrategias eficientes de preprocesamiento y representación, el manejo del desbalance de clases, la evaluación robusta del desempeño y la mitigación de \textit{domain shifts} derivados de diferencias en protocolos de tinción y adquisición \cite{campanella2019clinical}\cite{Tellez_2018}. Asimismo, la escasez de subtipos relevantes desde el punto de vista clínico plantea retos adicionales para el aprendizaje y la generalización de los modelos.\newline

En este contexto, el problema central que aborda este proyecto consiste en diseñar y evaluar un pipeline de aprendizaje débil capaz de clasificar imágenes histopatológicas prostáticas a partir de etiquetas \textit{slide-level}, preservando información diagnóstica relevante y garantizando una evaluación experimental reproducible. Particularmente, se investiga el uso de variantes de MIL basadas en mecanismos de atención, con el fin de analizar no solo su desempeño cuantitativo, sino también su estabilidad y potencial interpretativo.\newline

Finalmente, el proyecto adopta prácticas esenciales de reproducibilidad, tales como particiones estratificadas por paciente, control de semillas aleatorias y versionado del conjunto de parches, que permiten realizar comparaciones internas consistentes entre distintas configuraciones del modelo. Evaluaciones más extensas, como validaciones interinstitucionales o análisis cualitativos con múltiples patólogos, se reconocen como relevantes, pero se consideran fuera del alcance de este proyecto y se proponen como líneas de trabajo futuro.



%%%%%%%%%%%%%%%%%%%%%%
% OBJETIVOS
%%%%%%%%%%%%%%%%%%%%%%
\chapter{Objetivos}
% objetivos.tex
\section{Objetivo General}
Desarrollar un modelo de visión por computador a partir de técnicas de Aprendizaje con Múltiples Instancias para la clasificación débilmente supervisada de imágenes de histopatología de cáncer de próstata, con el fin de maximizar la precisión diagnóstica y facilitar su interpretación clínica.

\section{Objetivos Específicos}
\begin{enumerate}
    \item Implementar técnicas de procesamiento de imágenes de histopatología prostática, incluyendo corrección de color, eliminación de artefactos y normalización de tinciones.
    
    \item Implementar modelos de visión por computador basados en Aprendizaje de Instancias Múltiples con Deep Learning para la clasificación de Whole Slide Images.
    
    \item Evaluar el rendimiento de los modelos desarrollados mediante métricas estándar (AUC, precisión, recall, F1-score y coeficiente Kappa) con validación cruzada.

\end{enumerate}

%%%%%%%%%%%%%%%%%%%%%%
% MARCO TEORICO
%%%%%%%%%%%%%%%%%%%%%%
\chapter{Marco Teórico}
\section{Términos clave}
% ====================== TÉRMINOS CLAVE ======================

\noindent
A continuación se presentan las definiciones de los términos centrales utilizados a lo largo de este trabajo, con el fin de establecer una base conceptual precisa y homogénea que facilite la comprensión de los capítulos posteriores.

\begin{enumerate}

    \item \textbf{Bag e instancia:} En el paradigma de \textit{Aprendizaje de Instancias Múltiples} (Multiple Instance Learning, MIL), un \textit{bag} corresponde a un conjunto de instancias individuales que no cuentan con etiquetas propias, pero que están asociadas colectivamente a una etiqueta de nivel superior \cite{carbonneau2018multiple}. Formalmente, el aprendizaje se realiza a partir de pares $(\mathcal{X}_i, y_i)$, donde $\mathcal{X}_i$ denota el conjunto de instancias y $y_i$ la etiqueta del bag. En el contexto de la histopatología digital, cada \textit{Whole Slide Image} (WSI) se modela como un bag, mientras que los parches extraídos de dicha imagen constituyen las instancias. Esta formulación resulta especialmente adecuada en escenarios clínicos reales, donde las anotaciones detalladas a nivel de parche suelen ser inexistentes o costosas de obtener.

    \item \textbf{Técnicas de agregación:} Corresponden a los mecanismos mediante los cuales la información proveniente de múltiples instancias dentro de un bag es combinada para producir una predicción global. Estas técnicas definen explícitamente la función de decisión del modelo MIL. Entre los enfoques más comunes se encuentran los esquemas clásicos de \emph{pooling}, como \emph{mean pooling} y \emph{max pooling}, que actúan como líneas base, así como métodos más avanzados basados en mecanismos de atención, los cuales aprenden a ponderar dinámicamente la contribución de cada instancia en función de su relevancia para la tarea de clasificación \cite{ilse2018attention, li2021dual, GADERMAYR2024102337}.

    \item \textbf{Supervisión débil:} Hace referencia a escenarios de aprendizaje en los que la información de etiquetado es incompleta, imprecisa o disponible únicamente a un nivel de granularidad superior, en contraste con la supervisión totalmente anotada a nivel de instancia \cite{zhou2018brief, campanella2019clinical, GADERMAYR2024102337}. En el caso de la clasificación de WSIs, las etiquetas suelen estar disponibles únicamente a nivel de lámina completa, lo que impone restricciones significativas al entrenamiento supervisado tradicional y motiva el uso de enfoques MIL como una solución viable y clínicamente realista.

    \item \textbf{Mecanismo de atención:} Es una estrategia de agregación que permite al modelo aprender pesos diferenciados para cada instancia dentro de un bag, reflejando su contribución relativa a la predicción final. En el contexto de MIL, los mecanismos de atención no solo mejoran la robustez del modelo frente a ruido y heterogeneidad intra-bag, sino que además proporcionan interpretabilidad intrínseca al evidenciar qué subconjuntos de instancias resultan más informativos para la decisión del modelo, lo cual es particularmente relevante en aplicaciones médicas \cite{ilse2018attention, lu2021data}.

    \item \textbf{Heterogeneidad intra-lámina:} Describe la variabilidad estructural, morfológica y semántica presente entre los parches que componen una misma WSI. Esta heterogeneidad puede deberse a la coexistencia de tejido sano, regiones tumorales de distinta agresividad, artefactos de preparación o zonas no informativas, y constituye uno de los principales desafíos en la clasificación automática de imágenes histopatológicas \cite{GADERMAYR2024102337}.

    \item \textbf{Embeddings o representaciones profundas:} Son vectores de características de dimensión reducida que codifican información semántica relevante extraída por redes neuronales convolucionales (CNNs) entrenadas como extractores de características. En este trabajo, cada parche es representado mediante un embedding en un espacio latente, lo que permite desacoplar la etapa de extracción de características de la etapa de agregación y clasificación basada en MIL \cite{litjens2017survey}.

    \item \textbf{ISUP y escala de Gleason:} La \emph{International Society of Urological Pathology} (ISUP) define un sistema de gradación del cáncer de próstata basado en la evaluación de patrones arquitecturales glandulares observados en tinciones hematoxilina-eosina (H\&E) \cite{epstein2016gleason, SILVARODRIGUEZ2020105637}. La escala de Gleason asigna puntuaciones que reflejan la agresividad tumoral, mientras que las categorías ISUP agrupan dichas puntuaciones en clases clínicamente significativas. En este trabajo, estas categorías son reformuladas en un esquema binario con el objetivo de abordar una tarea de clasificación alineada con decisiones diagnósticas prácticas y con la disponibilidad de datos.

    \item \textbf{Coeficiente de concordancia de Cohen (Kappa):} Es una métrica estadística utilizada para cuantificar el grado de acuerdo entre dos clasificadores discretos, corrigiendo por la probabilidad de coincidencia atribuible al azar. Esta medida se define como
    \[
        \kappa = \frac{p_o - p_e}{1 - p_e},
    \]
    donde \(p_o\) es la proporción de acuerdos observados y \(p_e\) es la proporción de acuerdos esperados por azar. El coeficiente Kappa toma valores entre \(-1\) y \(1\), donde valores cercanos a 1 indican un alto nivel de acuerdo, valores cercanos a $0$ sugieren acuerdo consistente con el azar, y valores negativos representan acuerdo menor al esperado por azar \cite{mchugh2012interrater}. Esta métrica se emplea frecuentemente en estudios clínicos y de clasificación para evaluar la concordancia entre predicciones automatizadas y anotaciones de referencia.

\end{enumerate}


\section{Cáncer de próstata y escala de Gleason}
El cáncer de próstata representa una de las principales causas de morbilidad y mortalidad en la población masculina a nivel global, lo que sitúa a su diagnóstico y pronóstico como prioridades críticas en la salud pública. En la práctica clínica actual, estas tareas dependen fundamentalmente de la evaluación histopatológica, donde el examen de biopsias mediante tinciones convencionales de hematoxilina y eosina (H\&E) permite la aplicación de la escala de Gleason. Este sistema de gradación, basado en el análisis de los patrones arquitecturales de las glándulas prostáticas, constituye el referente para clasificar la agresividad tumoral y definir el riesgo clínico del paciente \cite{litjens2017survey} \cite{Brett_2016}.\newline

Sin embargo, la transición hacia una medicina de precisión exige integrar este dominio clínico con las capacidades de la ciencia de datos. Debido a que la interpretación de estas muestras genera grandes volúmenes de información visual de alta complejidad, se hace indispensable el uso de técnicas computacionales avanzadas. En este sentido, la presente sección revisa los conceptos fundamentales y los marcos metodológicos de aprendizaje profundo que permiten modelar estos datos provenientes de imágenes digitales, facilitando así una transición coherente desde la observación patológica tradicional hacia soluciones de diagnóstico asistido por computadora.\newline

Ahora bien, la asignación del puntaje de Gleason presenta una variabilidad interobservador significativa. Esta subjetividad resulta particularmente crítica en la diferenciación entre patrones limítrofes, como Gleason 3 y Gleason 4, al igual que en la identificación de subtipos histológicos asociados a peor pronóstico, entre ellos el patrón cribiforme \cite{SILVARODRIGUEZ2020105637}. Dicho patrón se caracteriza por estructuras glandulares con arquitectura tipo criba y se asocia a mayor agresividad tumoral y riesgo de recurrencia, lo que tiene implicaciones clínicas directas en la toma de decisiones terapéuticas.\newline

Esta variabilidad diagnóstica resalta la necesidad de herramientas computacionales que contribuyan a mejorar la reproducibilidad y consistencia de la evaluación histopatológica, constituyendo así un punto de partida natural para la aplicación de metodologías de aprendizaje automático sobre imágenes histológicas digitalizadas.

\begin{figure}[h]
    \centering
    \includegraphics[width=0.4\textwidth]{images/Four-basic-architectural-patterns-of-Gleason-pattern-4-A-Cribriform-glands-B-Fused_W640.jpg}
    % \includegraphics[width=0.30\textwidth]{images/Example-of-a-cribriform-pattern_W640.jpg}
    % \includegraphics[width=0.35\textwidth]{images/Gleason grade 4 patterns and intraductal carcinoma.jpg}
    % \includegraphics[width=0.35\textwidth]{images/cancers-14-03041-g001.png}
    \caption{Ejemplos representativos de patrones cribiformes (Gleason 4) en tejido prostático \cite{SILVARODRIGUEZ2020105637}.}
    \label{fig:cribriform_patterns}
\end{figure}


\section{Histopatología digital y Whole Slide Images (WSI)}
La histopatología digital, impulsada por el escaneo de muestras histológicas, ha revolucionado el análisis de tejidos al permitir la generación Whole Slide Images (WSI). Estas imágenes de resolución gigapíxel, como se puede apreciar en la figura \ref{fig:EjemploSuperResolucionWSI}, conservan la arquitectura tisular completa y posibilitan el análisis computacional a gran escala, prometiendo asistir a los patólogos al reducir el error humano y estandarizar los criterios de evaluación \cite{campanella2019clinical} \cite{lu2021data}.\newline

No obstante, el uso de las WSI plantea desafíos técnicos significativos debido a su tamaño extremo y la heterogeneidad visual inherente al tejido. Para hacer viable su procesamiento, estas imágenes se dividen en pequeños bloques o parches. Esta segmentación, sin embargo, introduce nuevos retos, como la variabilidad de tinción entre laboratorios, la presencia de artefactos de preparación y la complejidad morfológica intra e interpaciente \cite{Tellez_2018}.\newline

Ante esta restricción, el Aprendizaje de Instancias Múltiples surge como una alternativa particularmente atractiva, la cual permite entrenar modelos con etiquetas globales a nivel de la imagen (slide-level), reduciendo la dependencia de anotaciones exhaustivas por parte de expertos. Por lo tanto, cualquier pipeline reproducible bajo este esquema debe considerar cuidadosamente las decisiones de preprocesamiento, como la normalización de tinción y el filtrado por porcentaje de tejido, asegurando estrategias de muestreo que se adapten a la heterogeneidad del WSI. \cite{campanella2019clinical} \cite{lu2021data}.

\begin{figure}[h]
    \hspace*{4.5cm} % ajusta el valor según lo necesites (cm, mm, etc.)
    \includegraphics[scale=0.2]{images/Ejemplo de Super-Resolucion WSI.png}
    \caption{Ejemplo de imagen de super-resolución WSI \cite{Le_2023}}
    \label{fig:EjemploSuperResolucionWSI}
\end{figure}

\section{Redes neuronales convolucionales como extractores de características}
Las redes neuronales convolucionales (CNN) se han establecido como el paradigma dominante en la visión por computadora debido a su capacidad intrínseca para aprender jerarquías de características espaciales y patrones visuales complejos \cite{Géron_2022}. En el contexto de la histopatología digital, las CNN se emplean fundamentalmente como extractores de características a nivel de parche, lo cual es vital para procesar las imágenes de diapositivas completas (WSI) dada su gran escala.\newline

Cada parche es procesado individualmente por la CNN para generar una representación vectorial (embedding) que encapsula la información morfológica relevante, como las texturas, estructuras glandulares y patrones tisulares. Estas representaciones permiten reducir la subjetividad inherente a la interpretación visual humana y constituyen la entrada fundamental para los modelos MIL \cite{lu2021data} \cite{HEZI2025108513}.\newline

A continuación, se presenta la figura \ref{fig:cnn_arch} que ilustra la arquitectura típica de una CNN.\newpage

\begin{figure}[h]
    \centering
    \includegraphics[scale=0.3]{images/CNN_Typical_Arch.png}
    \caption{Arquitectura típica de una red neuronal convolucional \cite{Géron_2022}}
    \label{fig:cnn_arch}
\end{figure}

Una estrategia comúnmente adoptada es el uso de transfer learning con arquitecturas pre-entrenadas en grandes bases de datos como ImageNet. Modelos como ResNet o EfficientNet son adaptados para la tarea de clasificación histopatológica, lo cual acelera la convergencia del entrenamiento y mejora la generalización en conjuntos de datos de tamaño moderado \cite{lu2021data} \cite{he2016deep} \cite{tan2019efficientnet}. Para estabilizar el proceso de aprendizaje y mitigar el sobreajuste, es crucial incorporar técnicas de regularización y estrategias de aumentación de datos diseñadas específicamente para el dominio histopatológico.\newline

En el marco del Aprendizaje de Instancias Múltiples (MIL), las características vectoriales extraídas por la CNN de cada parche son posteriormente agregadas por módulos especializados, como los mecanismos de atención o el pooling jerárquico. Esto permite generar una predicción a nivel de la lámina completa, eliminando la necesidad de un etiquetado exhaustivo de cada parche y consolidando el pipeline de análisis de WSI.

\section{Aprendizaje de Instancias Múltiples (MIL)}
El \textbf{Aprendizaje de Instancias Múltiples (MIL)} se presenta como una solución para escenarios en los que las etiquetas están disponibles únicamente a nivel de muestra completa, o \textbf{“bag”}. En este contexto, una WSI es interpretada como un \textit{bag} de parches, y la tarea del modelo consiste en inferir la etiqueta global a partir de las características de las instancias locales. Si bien las primeras formulaciones se basaban en estrategias de agregación simples, como el \textit{max pooling}, bajo la premisa de que la presencia de un solo parche positivo determinaba el diagnóstico, este enfoque ignoraba la heterogeneidad del tejido y las cruciales relaciones espaciales entre parches \cite{carbonneau2018multiple} \cite{ilse2018attention}.
\\
\\
Para clarificar este planteamiento, la Figura~\ref{fig:wsi_patches} muestra un esquema de una WSI dividida en parches (instancias). En MIL las etiquetas se asignan exclusivamente a nivel de \emph{bag} (WSI) y no a cada parche individual; esa es la diferencia clave frente al aprendizaje supervisado a nivel de instancia.
\newpage
\begin{figure}[ht!]
    \centering
    \includegraphics[width=0.65\textwidth]{images/Esquema_Ejemplo_MIL.png}
    \caption{Esquema: WSI dividida en parches (instancias). En MIL la etiqueta se proporciona a nivel de bag (WSI) y no a nivel de instancia (parche).}
    \label{fig:wsi_patches}
\end{figure}

La investigación reciente ha impulsado variantes más sofisticadas del Aprendizaje de Instancias Múltiples que buscan superar las limitaciones de las estrategias de agregación simples. Entre ellas destacan los \textbf{mecanismos de atención}, que asignan pesos diferenciados a cada parche según su relevancia para la predicción global \cite{ilse2018attention} \cite{campanella2019clinical}; el \textbf{pooling jerárquico}, orientado a capturar información en múltiples escalas y niveles de representación \cite{li2021dual} \cite{pinckaers2020streaming}; y los \textbf{modelos probabilísticos con regularización espacial}, diseñados para modelar la correlación entre parches vecinos y estimar la incertidumbre asociada a la predicción \cite{lu2021data} \cite{queiroz2021spatial}.\newline 

Estos avances no solo han mejorado la precisión diagnóstica del MIL, sino que también han potenciado su \textbf{explicabilidad clínica}, al permitir la identificación visual de las regiones histológicas más relevantes para el diagnóstico final.

\subsection{MIL con mecanismos de atención e interpretabilidad}
Los mecanismos de atención representan una evolución clave dentro del paradigma de Aprendizaje de Instancias Múltiples (MIL), al permitir que el modelo asigne pesos diferenciados a cada instancia en función de su contribución a la predicción global. A diferencia de las estrategias de agregación tradicionales, como el max o mean pooling, los modelos basados en atención aprenden de manera explícita qué regiones del tejido son más informativas para la tarea de clasificación \cite{campanella2019clinical} \cite{ilse2018attention} .\newline

Desde el punto de vista metodológico, la atención permite modelar la heterogeneidad intratumoral presente en las imágenes histopatológicas, capturando la contribución relativa de múltiples parches relevantes en lugar de depender de una única instancia dominante. Este enfoque resulta particularmente adecuado en escenarios de supervisión débil, donde únicamente se dispone de etiquetas a nivel de lámina completa (WSI) \cite{lu2021data}.\newline

Además de mejorar el rendimiento predictivo, los mecanismos de atención aportan una forma de interpretabilidad post-hoc al facilitar la identificación y visualización de las instancias que influyen de manera más significativa en la decisión del modelo. Esta propiedad es especialmente valiosa en el contexto clínico, donde la transparencia, la trazabilidad y la posibilidad de inspeccionar visualmente las regiones relevantes son factores críticos para la confianza y adopción de sistemas de apoyo al diagnóstico \cite{campanella2019clinical} \cite{rudin2019stop}.

\subsection{Extensiones recientes de MIL: Transformers y variantes de CLAM}
Adicionalmente a los modelos basados en atención descritos previamente, la literatura reciente ha explorado extensiones que integran componentes adicionales como módulos de \textbf{transformers} o variantes de agregación más sofisticadas, con el objetivo de capturar relaciones complejas entre las instancias que componen un bag. Estas propuestas buscan modelar de manera más explícita las posibles dependencias entre parches y enriquecer la representación del bag a partir de interacciones más complejas entre las características extraídas de cada instancia.\newline

Un ejemplo prominente de estas aproximaciones es el uso de transformers dentro del marco MIL, donde mecanismos de \textit{self-attention} permiten modelar relaciones globales entre los parches de una WSI sin suponer independencia entre ellos. En el trabajo de Sens \textit{et al.}, se incorpora una pérdida de bag embedding para reforzar la capacidad discriminativa de un transformer aplicado a MIL, demostrando mejoras en conjuntos de datos histopatológicos estandarizados \cite{Sens2023_TransMIL}. De manera similar, Xiong \textit{et al.} proponen un esquema jerárquico de atención-guía para transformers que explota múltiples resoluciones dentro de una misma WSI, lo cual favorece la identificación de regiones discriminativas de manera más holística \cite{Xiong2023_TransformerMIL}.\newline

Paralelamente, enfoques como CLAM (Clustering-Constrained Attention MIL) han extendido la idea del mecanismo de atención al introducir ramas múltiples de atención y restricciones de agrupamiento para refinar las representaciones de instancias y mejorar la capacidad de clasificación en escenarios con etiquetas de nivel de bag \cite{Lu2021_CLAM}. Estas variantes han mostrado rendimiento prometedor en diversas tareas de clasificación, incluida la subtipificación y detección de regiones relevantes en tejidos, al tiempo que generan mapas de atención que pueden utilizarse como evidencia diagnóstica visual.\newline

No obstante, estas arquitecturas más complejas también implican compromisos metodológicos y prácticos importantes. Los modelos basados en transformers suelen requerir conjuntos de datos amplios o estrategias de pre-entrenamiento especializadas para evitar el sobreajuste, dado el elevado número de parámetros inherente a su estructura de \textit{self-attention}. Además, la implementación de múltiples ramas de atención o submódulos interdependientes incrementa la complejidad computacional y la demanda de recursos de memoria durante el entrenamiento y la inferencia.\newline

Dadas las restricciones de disponibilidad de datos, los objetivos de interpretabilidad y la necesidad de una evaluación reproducible en el contexto clínico, el presente trabajo se centra en modelos de atención MIL más simples, tales como ABMIL y SmABMIL, que proporcionan una interpretabilidad post-hoc directa y un balance práctico entre desempeño y complejidad. Esta elección permite una implementación robusta y eficiente sin incurrir en la sobrecarga computacional y de diseño que introducen los transformers o las variantes extendidas de CLAM.








% \begin{itemize}
%     \item Mecanismos de atención, que asignan pesos distintos a los parches según su relevancia. \newline
    
%     \item Pooling jerárquico, que captura información en múltiples escalas. \newline
    
%     \item Modelos probabilísticos con regularización espacial, que incorporan correlación entre parches vecinos y estiman la incertidumbre del modelo. 
% \end{itemize}

% Estos desarrollos permiten que MIL no solo mejore la precisión diagnóstica, sino también la explicabilidad clínica, al resaltar las regiones histológicas más relevantes.


\begin{comment}
El aprendizaje de instancias múltiples (MIL) surge como una extensión del aprendizaje supervisado, ideal para el análisis de imágenes de diapositivas completas (WSI) donde las anotaciones solo existen a nivel de la muestra (slide), y no de parche. En este paradigma, una bag (la WSI) se considera un conjunto de instancias (los parches). La formulación clásica establece que un bag se clasifica como positivo si contiene al menos una instancia positiva. Sin embargo, en histopatología esta asunción simple puede ser insuficiente, ya que ignora el contexto espacial y la heterogeneidad inherente a la muestra \cite{carbonneau2018multiple} \cite{ilse2018attention}. \newline

A diferencia de los enfoques tradicionales, los modelos de MIL han evolucionado significativamente para abordar estas limitaciones. Si bien se emplearon inicialmente esquemas de agregación simples como el max pooling, desarrollos más recientes han incorporado mecanismos de atención que ponderan la relevancia de cada parche y estructuras de pooling jerárquico que capturan información a diferentes escalas. De manera similar, los modelos probabilísticos con regularización espacial, como el propuesto en \cite{MoralesAlvarez2024}, modelan la dependencia entre los parches para estimar la incertidumbre del modelo y mejorar la interpretabilidad de las predicciones. Este enfoque es crucial, ya que estas arquitecturas buscan integrar la evidencia local y el contexto global para mejorar tanto la precisión como la confiabilidad de los resultados clínicos \cite{waqas2024survey}. \newline

En este contexto, las Redes Neuronales Convolucionales (CNN) operan como potentes extractores de características de cada parche, y sus representaciones vectoriales son posteriormente agregadas por el módulo MIL. Este pipeline conjunto consolida una metodología robusta que elimina la necesidad de un etiquetado exhaustivo a nivel de instancia, optimizando el flujo de trabajo en la patología digital. Un ejemplo de este esquema, enfocado en el mecanismo de atención, se ilustra en la Figura \ref{fig:mil_attention}.

\begin{figure}[h]
    \centering
    \includegraphics[width=0.7\textwidth]{images/Esquema_Ejemplo_MIL.png}
    \caption{Esquema general del aprendizaje de instancias múltiples con mecanismo de atención \cite{ilse2018attention}.}
    \label{fig:mil_attention}
\end{figure}
\end{comment}

\section{Métricas de evaluación y análisis}
La validación rigurosa de modelos en histopatología digital va más allá del simple desempeño numérico, requiriendo métricas que reflejen tanto la utilidad técnica como la relevancia clínica del sistema \cite{Géron_2022}. En este contexto, la evaluación se basa en un conjunto de indicadores clave, cada uno de los cuales ofrece una perspectiva diferente sobre el comportamiento del modelo:

\begin{itemize}
    \item \textbf{AUC (Área bajo la curva ROC):} evalúa la capacidad de discriminación del modelo para distinguir entre las clases positivas y negativas a través de todos los posibles umbrales de clasificación. Un valor de AUC cercano a 1 indica una excelente capacidad para rankear correctamente los casos.
    
    \item \textbf{Precisión:} mide la proporción de diagnósticos positivos correctos sobre el total de casos clasificados como positivos. En un escenario clínico, una alta precisión es crucial, ya que minimiza los falsos positivos que podrían llevar a tratamientos innecesarios.
    
    \item \textbf{Recall o sensibilidad:} cuantifica la proporción de casos positivos que el modelo identificó correctamente. Una alta sensibilidad es vital para evitar los falsos negativos, que representan el riesgo de no detectar una enfermedad en su etapa inicial.
    
    \item \textbf{F1-score:} representa la media armónica entre precisión y recall. Esta métrica es especialmente útil en conjuntos de datos desbalanceados, ya que proporciona un solo valor que resume el equilibrio entre evitar falsos positivos y falsos negativos.
    
    \item \textbf{Coeficiente Kappa de Cohen:} permite cuantificar el grado de acuerdo entre las predicciones del modelo y las etiquetas reales, corrigiendo el acuerdo que podría ocurrir por azar. A diferencia de la precisión simple, el coeficiente Kappa es particularmente relevante para comparar el desempeño del modelo con el grado de acuerdo interobservador entre especialistas humanos.
    
\end{itemize}
    
Más allá de las métricas numéricas, la interpretabilidad se erige como un pilar fundamental para la integración de estos sistemas en la práctica clínica. En este sentido, el uso de mecanismos de atención permite trascender la naturaleza de 'caja negra' de los modelos de aprendizaje profundo al identificar explícitamente las regiones histológicas que fundamentan la predicción. Complementariamente, la cuantificación de la incertidumbre actúa como un mecanismo de seguridad, permitiendo discriminar entre inferencias de alta confianza y casos ambiguos que exigen la intervención de un patólogo experto.\cite{Géron_2022}.


%%%%%%%%%%%%%%%%%%%%%%
% ESTADO DEL ARTE
%%%%%%%%%%%%%%%%%%%%%%
\section{Estado del Arte}
La investigación en aprendizaje débilmente supervisado sobre Whole Slide Images (WSI) ha avanzado rápidamente en los últimos años, motivada por la necesidad de reducir la carga de anotación experta y por el potencial clínico de modelos reproducibles y explicables. Varios trabajos han propuesto arquitecturas y estrategias que permiten agregar evidencia local (parches) en predicciones a nivel de muestra (bag), y han explorado alternativas para incorporar contexto espacial, medidas de incertidumbre y visualizaciones de interpretabilidad. \newline

Para contextualizar la presente investigación, a continuación se revisan los antecedentes más relevantes que abordan estas problemáticas. Dichos estudios justifican las decisiones metodológicas de este proyecto al ilustrar las principales tendencias y desafíos en el campo, sirviendo como base teórica para el diseño de la solución propuesta:

\subsection{CAMIL: channel attention-based multiple instance learning 
for whole slide image classification}
En este trabajo, los autores proponen una arquitectura denominada FA-MILNet (Feature Aggregation Multiple Instance Learning Network), orientada a abordar los desafíos del aprendizaje débilmente supervisado en la clasificación de Whole Slide Images (WSI). La propuesta surge como respuesta a las limitaciones de los enfoques MIL tradicionales, los cuales suelen perder información discriminativa al emplear estrategias de agregación simples como el max pooling y no modelan adecuadamente la heterogeneidad intralaminar.\newline

FA-MILNet introduce un esquema de agregación jerárquica asistida por atención, integrando información local y global mediante tres componentes principales: extracción de características basada en CNN, un módulo de atención para ponderar la relevancia de los parches y una rama dual de agregación. Esta arquitectura permite mejorar simultáneamente el desempeño predictivo y la interpretabilidad del modelo \cite{Mao2025_CAMIL}.\newline

Evaluado en conjuntos de datos de cáncer de colon, mama y ganglios linfáticos (Camelyon16, BACH y CRC), el modelo supera enfoques como ABMIL, CLAM y DSMIL en métricas como AUC y F1-score. Si bien el estudio no se enfoca en cáncer de próstata, su diseño modular y su capacidad para operar sin anotaciones patch-level lo convierten en un antecedente altamente pertinente para este proyecto, al validar el uso de mecanismos de atención como estrategia central para la agregación de información en escenarios de supervisión débil \cite{Mao2025_CAMIL}.\newline


% --------------------------------------------------------------------------

\subsection{A Probabilistic MIL Model with Spatial Regularization for Weakly Supervised Histopathology Image Analysis}
Este trabajo presenta un modelo de Aprendizaje de Instancias Múltiples probabilístico, basado en procesos gaussianos e integrado con un término de regularización espacial inspirado en el modelo de Ising. Se parte de la premisa de que los parches histológicamente cercanos presentan una alta correlación diagnóstica, por lo que modelar explícitamente estas dependencias espaciales puede mejorar la robustez del aprendizaje bajo supervisión débil \cite{MoralesAlvarez2024}.\newline

A diferencia de los modelos MIL determinísticos, esta propuesta permite estimar distribuciones de probabilidad sobre las etiquetas de las instancias, proporcionando una cuantificación explícita de la incertidumbre asociada a las predicciones. Esta característica resulta particularmente relevante en contextos clínicos, donde la fiabilidad del modelo y la identificación de casos ambiguos son aspectos críticos.\newline

A diferencia de los modelos MIL determinísticos, esta propuesta permite estimar distribuciones de probabilidad sobre las etiquetas de las instancias, proporcionando una cuantificación explícita de la incertidumbre asociada a las predicciones. Esta característica resulta particularmente relevante en contextos clínicos, donde la fiabilidad del modelo y la identificación de casos ambiguos son aspectos críticos \cite{MoralesAlvarez2024}.
% --------------------------------------------------------------------------

\subsection{Graph-based Multiple Instance Learning for Whole Slide Image Classification}
Aunque este enfoque ofrece ventajas claras en términos de interpretabilidad y robustez, su complejidad computacional y los requerimientos asociados a la modelación probabilística exceden el alcance experimental del presente proyecto. No obstante, el trabajo constituye un antecedente fundamental al evidenciar la importancia de incorporar información espacial y estimación de incertidumbre en modelos MIL aplicados a histopatología digital. \cite{Fourkioti2023}. \newline

En este estudio se propone una variante de MIL basada en grafos, donde cada parche de una WSI se modela como un nodo y las relaciones espaciales entre parches se representan mediante aristas en un grafo no dirigido. Sobre esta estructura se emplea una Graph Neural Network (GNN) que permite propagar información contextual entre regiones del tejido, fortaleciendo la capacidad del modelo para capturar patrones arquitectónicos globales. \newline

No obstante, a diferencia de los mecanismos de atención utilizados en el presente proyecto, los enfoques basados en grafos introducen una complejidad adicional tanto en la construcción de la representación como en el entrenamiento del modelo, lo que motivó la elección de arquitecturas más ligeras y directamente interpretables \cite{Fourkioti2023}.

% --------------------------------------------------------------------------

\subsection{Deep Recurrent Attention MIL for Histopathological Image Analysis}
El enfoque planteado en este artículo se distingue por combinar mecanismos de atención con redes neuronales recurrentes (RNN) dentro del marco de MIL. En lugar de tratar los parches como elementos independientes, los autores introducen una secuencia ordenada basada en la posición espacial de cada uno, la cual es procesada por una RNN con atención. Este diseño permite capturar no solo las características locales, sino también las relaciones de largo alcance y las dependencias contextuales más profundas a lo largo de toda la WSI \cite{Qu_Yang_Huang_Guo_Luo_Zhang_Wang_2024}. \newline

Gracias a esta arquitectura secuencial, el modelo demuestra un mejor rendimiento en la identificación de estructuras glandulares complejas, un desafío particular en escenarios con gran variabilidad morfológica entre pacientes. Además, el mecanismo de atención adaptativa complementa este diseño, facilitando interpretaciones más precisas del modelo al resaltar de forma dinámica las regiones más relevantes. La principal lección de este trabajo es que la secuencia y el orden de los parches, y no solo su presencia, pueden aportar información crucial. A diferencia de los enfoques basados en grafos que modelan la vecindad de forma explícita, este método explora una representación secuencial, lo que invita a considerar cómo la estructura de los datos de entrada puede ser modificada para extraer de manera más efectiva la información contextual \cite{Qu_Yang_Huang_Guo_Luo_Zhang_Wang_2024}.

% --------------------------------------------------------------------------

\subsection{Hierarchical Pooling in MIL for Histopathological Subtype Classification}
Este trabajo propone un esquema de agregación jerárquica dentro del paradigma MIL, en el cual los parches primero se agrupan en regiones intermedias según su proximidad o similitud morfológica. Posteriormente, se aplica un segundo nivel de pooling que combina la información de estas regiones para obtener la predicción final del bag. Esta estructura multinivel permite capturar patrones de organización tisular más amplios, que son clave en el diagnóstico de ciertos subtipos de cáncer. La metodología incorpora tanto técnicas de agrupamiento automático como atención regional, y ha mostrado un desempeño superior en conjuntos de datos con alta heterogeneidad intratumoral, al mantener el balance entre granularidad local y contexto estructural \cite{waqas2024survey}. \newline

Este antecedente es relevante porque, al igual que FA-MILNet, enfatiza la importancia de una agregación multinivel. Sin embargo, se distingue por su énfasis en la creación de ''grupos'' lógicos de parches antes de la agregación final, lo que es una estrategia diferente a la simple ponderación por atención. Por otro lado. la idea de agrupar parches basándose en la morfología o la proximidad podría ser una técnica poderosa a considerar para proyectos que buscan capturar la complejidad de la heterogeneidad tumoral \cite{waqas2024survey}.


\subsection{Síntesis crítica y posicionamiento del presente trabajo}
Propuesta de Redacción Académica (Nivel Maestría)
"La revisión del estado del arte evidencia que el Aprendizaje de Instancias Múltiples (MIL) aplicado a Whole Slide Images ha experimentado una evolución significativa, integrando mecanismos de atención, modelado espacial explícito y esquemas de agregación jerárquica. Estos avances han derivado en mejoras sustanciales tanto en el rendimiento predictivo como en la capacidad interpretativa, demostrando una eficacia particular en escenarios caracterizados por una supervisión débil y una elevada heterogeneidad morfológica..\newline

No obstante, la sofisticación de muchas de estas aproximaciones conlleva una complejidad computacional onerosa o depende de supuestos restrictivos sobre la estructura espacial de los datos, factores que suelen obstaculizar su adopción en entornos clínicos reales y comprometer su evaluación reproducible. Bajo esta premisa, el presente trabajo se sitúa en un punto de convergencia estratégico, priorizando un equilibrio sinérgico entre la precisión diagnóstica, la interpretabilidad post-hoc y la viabilidad experimental.\newline

En concreto, se adopta un marco de trabajo fundamentado en MIL con mecanismos de atención, el cual permite ponderar la contribución relativa de múltiples regiones de interés sin incurrir en la necesidad de anotaciones locales ni en la imposición de estructuras espaciales explícitas. Esta elección metodológica responde a la imperativa de desarrollar un pipeline reproducible y escalable, alineado con las restricciones de disponibilidad de datos en histopatología digital, y establece un cimiento robusto para futuras extensiones que busquen incorporar información topológica o modelos probabilísticos de mayor complejidad.


%%%%%%% Definición de Requisitos %%%%%%%%%%%%%
\chapter{Definición de Requisitos}
La definición de requisitos constituye una etapa fundamental en el desarrollo del presente proyecto, ya que permite establecer de manera explícita las capacidades, restricciones y criterios de validación del sistema propuesto. Estos requisitos se derivan directamente del análisis del problema, los objetivos planteados y los antecedentes revisados en el estado del arte, funcionando como un marco de referencia para evaluar la coherencia y el alcance de la solución desarrollada.\newline

Dado que el objetivo central de esta investigación es el diseño y evaluación de un modelo de MIL para el análisis de imágenes histopatológicas digitales de próstata bajo un esquema de supervisión débil, los requisitos se formulan desde una perspectiva metodológica y científica, y sirven como puente entre la fase de concepción del sistema y su posterior implementación y evaluación experimental.

\section{Alcance del sistema propuesto}
El sistema desarrollado en este proyecto tiene como finalidad analizar \textit{Whole Slide Images} (WSI) de biopsias de próstata mediante un enfoque de aprendizaje débilmente supervisado, con el objetivo de inferir etiquetas diagnósticas a nivel de muestra completa. En este contexto, cada WSI es modelada como un conjunto de parches (instancias), a partir de los cuales se extraen representaciones profundas que posteriormente son agregadas mediante un modelo MIL con mecanismos de atención.\newline

El alcance del sistema se limita al análisis computacional de imágenes histopatológicas previamente digitalizadas y no contempla procesos clínicos como la adquisición de muestras, la digitalización de las láminas ni la validación clínica prospectiva de los resultados. Asimismo, el sistema está diseñado como una herramienta de apoyo a la investigación y no como un producto clínico listo para su despliegue en entornos hospitalarios.


\section{Requisitos funcionales}
Desde el punto de vista funcional, el sistema debe satisfacer las siguientes capacidades fundamentales:
\begin{itemize}
    \item \textbf{Procesamiento de imágenes de gran escala:} El sistema debe permitir el análisis eficiente de WSIs de alta resolución, descomponiéndolas en parches que capturen información morfológica relevante sin requerir anotaciones a nivel de instancia.

    \item \textbf{Pipeline de preprocesamiento:} Se requiere una etapa de preprocesamiento orientada a la generación de parches homogéneos y filtrados, eliminando regiones no informativas como fondo, ruido o artefactos de preparación.

    \item \textbf{Extracción de características:} El sistema debe incorporar un mecanismo basado en redes neuronales convolucionales (CNN) preentrenadas para obtener representaciones vectoriales (\textit{embeddings}) discriminativas de cada parche. Estas representaciones constituyen la entrada principal del modelo MIL encargado de inferir la predicción a nivel de WSI.

    \item \textbf{Mecanismos de atención:} Es necesaria la integración de mecanismos de atención que asignen pesos diferenciales a las instancias en función de su relevancia diagnóstica, permitiendo identificar las regiones del tejido que contribuyen de manera dominante a la decisión del modelo.

    \item \textbf{Generación de métricas y visualizaciones:} El sistema debe generar métricas de evaluación cuantitativas a nivel de lámina completa (slide), así como visualizaciones que reflejen la distribución de la atención sobre la WSI, facilitando el análisis posterior de los resultados.
\end{itemize}



\section{Requisitos no funcionales}
En términos de atributos de calidad y operatividad, el sistema debe garantizar:
\begin{itemize}
    \item \textbf{Reproducibilidad:} Los experimentos deben ser replicables mediante el uso de semillas aleatorias controladas, configuraciones explícitas y una organización estructurada del código y los experimentos.
    
    \item \textbf{Escalabilidad:} El sistema debe poder manejar WSIs con un número elevado de parches, sin que ello implique un crecimiento prohibitivo en los tiempos de cómputo, mediante un diseño modular que desacople la extracción de características de la agregación MIL.
    
    \item \textbf{Interpretabilidad:} Dado el dominio clínico del problema, el sistema debe proporcionar mecanismos que permitan analizar y justificar las decisiones del modelo, evitando enfoques de tipo ``caja negra''. Si bien no se busca una interpretabilidad causal estricta, sí se requiere una interpretabilidad post-hoc basada en mecanismos de atención y visualización de regiones relevantes.
    
    \item \textbf{Compatibilidad:} El sistema debe ser compatible con infraestructuras estándar en investigación  de ciencia de datos, como entornos basados en GPU y \textit{frameworks} de aprendizaje profundo ampliamente adoptados.
\end{itemize}


\section{Restricciones del proyecto}
El desarrollo del sistema se encuentra condicionado por diversas restricciones inherentes tanto al dominio clínico como al contexto académico de la investigación. Estas limitaciones se describen a continuación:
\begin{enumerate}
    \item \textbf{Supervisión débil:} La disponibilidad de etiquetas se limita al nivel de WSI, lo que impide el uso de enfoques supervisados a nivel de parche.
    
    \item \textbf{Heterogeneidad morfológica:} La alta variabilidad intratumoral en las biopsias de próstata dificulta la agregación simple de información y refuerza la necesidad de mecanismos de atención y modelos capaces de manejar dicha variabilidad.
    
    \item \textbf{Recursos computacionales:} El tamaño de las WSIs y la cantidad de parches generados imponen limitaciones en términos de memoria y tiempo de entrenamiento, lo que obliga a adoptar estrategias de muestreo, reducción de dimensionalidad o separación de etapas dentro del pipeline.
    
    \item \textbf{Validación clínica:} Al tratarse una investigación académica, el proyecto está limitado en términos de validación clínica directa, por lo que los resultados deben ser interpretados como evidencia experimental y no como conclusiones clínicas definitivas.
\end{enumerate}




\section{Criterios de validación}
Los criterios para validar el sistema se definen bajo los siguientes parámetros:
\begin{itemize}
    \item \textbf{Evaluación cuantitativa:} El desempeño del modelo debe ser evaluado mediante métricas cuantitativas apropiadas para problemas de clasificación a nivel de WSI, tales como accuracy, AUC y F1-score, permitiendo una comparación objetiva entre diferentes configuraciones del modelo.
    
    \item \textbf{Estabilidad y generalización:} Se debe evaluar la consistencia del modelo frente a distintas particiones de los datos (validación cruzada) con el fin de analizar su capacidad de generalización. Este criterio es especialmente relevante dado el tamaño limitado de los conjuntos de datos histopatológicos y la alta variabilidad entre muestras.
    
    \item \textbf{Coherencia histopatológica:} Como criterio cualitativo, los mapas de atención generados deben ser coherentes desde un punto de vista histopatológico, es decir, que las regiones destacadas por el modelo correspondan a áreas con características morfológicas plausibles según el conocimiento experto. Este análisis no pretende sustituir la evaluación clínica, pero sí aportar evidencia sobre la consistencia y utilidad interpretativa del enfoque propuesto.

\end{itemize}

\chapter{Diseño del Pipeline}
% ======================
% DISEÑO DEL PIPELINE
% ======================

\section{Diseño del Pipeline Experimental}

Este capítulo describe de manera detallada el diseño del sistema de clasificación de WSIs mediante Aprendizaje de Instancias Múltiples (MIL). El objetivo es presentar un flujo de procesamiento de datos, extracción de características y modelado que permita una evaluación reproducible del problema médico de clasificación binaria de tejido prostático.

El diseño sigue una metodología incremental basada en Sprints, la cual facilita la organización del desarrollo, la gestión de riesgos y la trazabilidad entre los objetivos del proyecto y la implementación computacional.

\subsection{Sprint 0: Infraestructura, Entorno y Reproducibilidad}

El propósito de este Sprint fue establecer un entorno de desarrollo robusto, escalable y reproducible, preparado para procesar imágenes de resolución gigapíxel sin comprometer la consistencia de los resultados.

\subsubsection{Infraestructura Computacional}

Se configuró un entorno de ejecución basado en instancias con aceleración por GPU (p. ej., NVIDIA T4/V100) para cumplir con los requerimientos computacionales del procesamiento de WSIs. La compatibilidad con CUDA se verificó mediante pruebas de ejecución en PyTorch \cite{paszke2019pytorch}.

\subsubsection{Gestión de Dependencias}

Se instaló el conjunto de librerías necesarias para manipulación de imágenes médicas y machine learning:

\begin{itemize}
    \item \textbf{openslide-python:} Lectura eficiente de pirámides de imágenes WSI sin saturar la memoria principal.
    \item \textbf{Albumentations:} Generación de transformaciones de datos para aumentación durante entrenamiento.
    \item \textbf{OpenCV / scikit-image:} Procesamiento de imágenes y filtrado de contenido útil.
\end{itemize}

\subsubsection{Reproducibilidad Científica}

Se estableció un protocolo de reproducibilidad a nivel de código y entrenamiento con el fijado de una semilla global:

\begin{itemize}
    \item Función de fijación de semilla (\texttt{set\_seed(42)}) para controlar la aleatoriedad en \texttt{numpy}, \texttt{random} y \texttt{torch}.
    \item Uso de entornos virtuales y definición explícita de versiones de librerías para garantizar reproducibilidad completa de resultados.
\end{itemize}

\subsection{Sprint 1: Preprocesamiento y Refinamiento de Datos}

El objetivo de este Sprint fue transformar los datos crudos (WSIs + anotaciones) en un conjunto de tensores limpios, coherentes y libres de inconsistencias, listos para extracción de características y modelado.

\subsubsection{Ajuste de Etiquetado Binario}

Se detectó ambigüedad en la lógica original de lectura de etiquetas desde archivos Excel de particiones. Para resolverlo:

\begin{itemize}
    \item \textbf{Clase 0 (Negativo):} WSIs etiquetadas como 'NC' (No Cancer).
    \item \textbf{Clase 1 (Positivo):} WSIs con Gleason Score $\ge 6$ (G3, G4, G5), acorde con definiciones clínicas \cite{epstein2016gleason, SILVARODRIGUEZ2020105637}.
\end{itemize}

Esta binarización elimina ruido de clases intermedias mal etiquetadas y permite que la definición de la verdad terreno (ground truth) sea consistente para el entrenamiento y la evaluación.

\subsubsection{Filtrado de Tejido Relevante}

Debido al gran tamaño y variabilidad de las WSI, se implementó un enfoque de segmentación basado en umbrales de color:

\begin{itemize}
    \item \textbf{Espacio de Color HSV:} Se calcula una máscara de saturación ($S>0.05$) para separar tejido de fondo.
    \item \textbf{Umbral de Descarte:} Un parche se retiene sólo si contiene $\ge 50\%$ de tejido útil, reduciendo la inclusión de parches vacíos o artefactos.
\end{itemize}

Este filtrado optimiza el entrenamiento al garantizar que sólo parches relevantes sean procesados en etapas posteriores.

\subsection{Sprint 2: Ingeniería de Características}

Este Sprint se centró en la reducción de dimensionalidad y la representación de cada parche mediante vectores numéricos densos (embeddings), evitando la fuga de información entre conjuntos.

\subsubsection{Extracción de Características (Feature Extraction)}

Se empleó un modelo de referencia pre-entrenado para obtener representaciones de cada parche:

\begin{itemize}
    \item \textbf{Backbone:} ResNet50 pre-entrenado en ImageNet, truncado eliminando la última capa totalmente conectada.
    \item \textbf{Entrada:} Tensores de dimensión $(3, 512, 512)$.
    \item \textbf{Salida:} Vectores de características de dimensión $(2048)$ por parche.
\end{itemize}

Este enfoque de transfer learning facilita la captura de patrones relevantes sin requerir entrenamiento desde cero \cite{Litjens2017Survey}.

\subsubsection{Prevención de Fuga de Datos (Group-aware Split)}

Para evitar que parches del mismo paciente se distribuyan entre conjuntos de entrenamiento y prueba, se empleó un esquema de validación estratificada por paciente:

\begin{itemize}
    \item Se usó \textbf{GroupKFold} para generar particiones donde todos los parches de un mismo patient\_id permanecen en el mismo fold.
    \item Esto garantiza una evaluación honesta y clínicamente válida del modelo.
\end{itemize}

\subsubsection{Optimización de Almacenamiento}

Para mejorar la eficiencia computacional:

\begin{itemize}
    \item Se serializaron los embeddings en formatos binarios (.npy, .pt) para acelerar la carga durante entrenamiento y validación.
    \item Esto evita repetidas decodificaciones de imágenes en disco durante cada ciclo de entrenamiento.
\end{itemize}

\subsection{Sprint 3: Modelado MIL y Validación}

Este Sprint corresponde al núcleo del pipeline: la definición del modelo MIL basado en atención y su entrenamiento riguroso.

\subsubsection{Arquitectura de Atención (Gated Attention MIL)}

Se implementó una variante de MIL con atención que:

\begin{itemize}
    \item Procesa bolsas de instancias de tamaño variable.
    \item Asigna pesos de importancia a cada parche.
    \item Facilita interpretabilidad al evidenciar qué parches son más determinantes para la predicción final \cite{ilse2018attention}.
\end{itemize}

La elección de un mecanismo de atención específico responde a la necesidad de interpretabilidad clínica y coherencia con la supervisión débil.

\subsubsection{Entrenamiento y Validación}

\begin{itemize}
    \item Se entrenó el modelo bajo una estrategia de validación cruzada estratificada por paciente (GroupKFold).
    \item La evaluación se realizó a nivel de WSI/bag, utilizando métricas coherentes con lo definido en el Marco Teórico.
\end{itemize}

\subsection{Sprint 4: Interpretación y Mapas de Atención}

La generación de mapas de atención permite:

\begin{itemize}
    \item Identificar regiones de tejido que más aportan a la decisión.
    \item Relacionar las predicciones del modelo con hallazgos clínicamente relevantes.
    \item Validar la interpretabilidad del modelo en bases de datos reales, como SICAPv2.
\end{itemize}

La interpretación de estos mapas se complementa con análisis cuantitativos y visuales presentados en el capítulo de Resultados y Análisis.

\subsection{Sprint 5: Puesta a Punto y Ajuste Final}

Este Sprint se dedicó a:

\begin{itemize}
    \item Ajuste de hiperparámetros mediante búsqueda estructurada.
    \item Evaluación del impacto de preprocesamientos alternativos.
    \item Validación con diferentes particiones de grupos para confirmar robustez de desempeño.
\end{itemize}

\section{Resumen del Diseño del Pipeline}

Este pipeline combina una ingeniería de características cuidadosamente diseñada y un modelo MIL con atención interpretativa. La estrategia de validación, la prevención de fuga de datos y la serialización de embeddings aseguran resultados reproducibles y clínicamente relevantes.

\subsection{Matriz de trazabilidad entre objetivos y requisitos}

Con el fin de garantizar la coherencia entre los objetivos específicos del proyecto y su implementación técnica, se construyó una matriz de trazabilidad que relaciona cada objetivo con los requisitos funcionales correspondientes y el componente del pipeline en el que estos son abordados. Esta matriz permite verificar que todas las decisiones metodológicas y de diseño experimental están directamente alineadas con los objetivos planteados.

\begin{table}[h!]
    \centering
    \caption{Matriz de trazabilidad entre objetivos específicos, requisitos funcionales y componentes del pipeline}
    \label{tab:matriz_trazabilidad}
    \begin{tabular}{|p{4cm}|p{4cm}|p{4cm}|}
        \hline
        \textbf{Objetivo Específico} & \textbf{Requisito Funcional} & \textbf{Componente del Pipeline} \\
        \hline
        OE1: Preprocesamiento y estructuración de datos. & RF-01, RF-02, RF-03 & Sprint 0 y Sprint 1 \\
        \hline
        OE2: Diseño de modelos de aprendizaje profundo. & RF-04, RF-05, RF-06 & Sprint 2 y Sprint 3 \\
        \hline
        OE3: Validación y evaluación diagnóstica. & RF-07 & Sprint 4 \\
        \hline
    \end{tabular}
\end{table}


%%%%%%% CONCEBIR %%%%%%%%%%%%%
\chapter{Metodología Experimental y Resultados}

Este capítulo describe de manera detallada la metodología experimental empleada en el desarrollo del proyecto, así como el diseño del pipeline computacional utilizado para el análisis de imágenes histopatológicas prostáticas. El objetivo principal es garantizar la trazabilidad entre los requisitos definidos previamente, las decisiones metodológicas adoptadas y los resultados obtenidos, proporcionando un marco reproducible, coherente y clínicamente relevante para la evaluación del modelo propuesto.\newline

La estructura del capítulo sigue una progresión lógica desde la descripción general del pipeline experimental hasta el detalle de cada una de sus etapas, incluyendo la preparación de los datos, la extracción de características profundas, el modelado mediante Aprendizaje de Instancias Múltiples (MIL) y el protocolo de evaluación empleado para cuantificar el desempeño del sistema.

%-------------------------------------------------------------

\section{Visión general del pipeline experimental}

El \textit{pipeline} desarrollado se fundamenta en un diseño modular, desacoplado y determinista, concebido para gestionar la complejidad computacional inherente a las imágenes histopatológicas de gigapíxeles y, simultáneamente, garantizar rigor metodológico y trazabilidad diagnóstica. Cada etapa del flujo de trabajo se implementó como un módulo independiente, con interfaces claramente definidas, lo que facilita tanto la reproducibilidad experimental como el análisis sistemático de cada componente.\newline

Con el fin de asegurar la consistencia de los resultados, se estableció un control estricto de la aleatoriedad mediante la fijación de semillas globales, garantizando que la inicialización de los modelos y la partición de los datos en los distintos \textit{folds} produzcan resultados idénticos ante ejecuciones sucesivas bajo las mismas condiciones experimentales.\newline

Para mejorar la legibilidad del flujo de trabajo completo y resaltar la separación conceptual entre las etapas de preprocesamiento y extracción de representaciones, por un lado, y las fases de modelado MIL y evaluación, por otro, el pipeline experimental se presenta en dos bloques complementarios. Como se ilustra en la Figura~\ref{fig:pipelineProcesamiento} y la Figura~\ref{fig:pipelineModelado}, el proceso completo se organiza en siete fases principales.


\newpage
\begin{figure}[ht]
    \centering
    \includegraphics[width=0.8\textwidth]{images/PipelineDiag1.png}
    \caption{Diagrama integral del \textit{pipeline} experimental propuesto: procesamiento y extracción de representaciones.}
    \label{fig:pipelineProcesamiento}
\end{figure}

\begin{figure}[h!]
    \centering
    \includegraphics[width=0.8\textwidth]{images/PipelineDiag2.png}
    \caption{Diagrama integral del \textit{pipeline} experimental propuesto: modelado MIL y evaluación.}
    \label{fig:pipelineModelado}
\end{figure}

\newpage
\begin{enumerate}
    \item \textbf{Modelado de la entrada y supervisión débil:} Cada \textit{Whole Slide Image} (WSI) se conceptualiza como una entidad compuesta por múltiples instancias locales (parches). La etiqueta diagnóstica, derivada del grado de Gleason, se asocia exclusivamente al nivel global de la lámina, evitando el uso de anotaciones manuales a nivel de parche. Esta formulación refleja fielmente las condiciones reales de la práctica clínica y define un escenario de supervisión débil.

    \item \textbf{Segmentación, filtrado y generación de parches:} Las WSIs del conjunto SICAPv2 se subdividen en parches de H\&E mediante un proceso de \textit{tiling}. Se aplica un filtrado automático basado en contenido tisular para descartar regiones dominadas por fondo o artefactos, garantizando que el modelo procese exclusivamente información morfológica relevante.

    \item \textbf{Codificación semántica mediante transferencia de aprendizaje:} Cada parche filtrado se transforma en un vector de características de dimensión fija utilizando una red convolucional profunda preentrenada (ResNet-50). En esta etapa no se realiza clasificación local; la CNN actúa únicamente como extractor de representaciones semánticas.

    \item \textbf{Estructuración de bolsas MIL y separación clínica:} Los \textit{embeddings} generados se agrupan según el identificador de la lámina (\texttt{slide\_id}) para conformar las bolsas MIL. Se garantiza la separación estricta por paciente durante la validación cruzada, evitando cualquier forma de fuga de información.

    \item \textbf{Modelado MIL con mecanismos de atención:} El núcleo predictivo se basa en un modelo \textit{Attention-based MIL}, capaz de asignar pesos de importancia a cada instancia dentro de una bolsa. Este mecanismo permite que las regiones patológicamente relevantes contribuyan en mayor medida a la representación global de la WSI.

    \item \textbf{Inferencia diagnóstica a nivel de slide:} La representación agregada de la lámina se proyecta a través de una capa totalmente conectada para obtener una probabilidad diagnóstica global, correspondiente a la presencia o ausencia de malignidad.

    \item \textbf{Validación robusta y evaluación del desempeño:} El desempeño del sistema se evalúa exclusivamente a nivel de WSI mediante validación cruzada por paciente, reportando métricas alineadas con el contexto clínico y el paradigma MIL.
\end{enumerate}

%-------------------------------------------------------------

\section{Descripción del conjunto de datos}
El desarrollo experimental de este trabajo se sustenta en la base de datos pública \textbf{SICAPv2}, un referente en el estudio del adenocarcinoma de próstata mediante patología digital. El conjunto de datos está compuesto por $155$ WSIs correspondientes a biopsias prostáticas provenientes de $95$ pacientes únicos, adquiridas mediante escáneres de alta resolución y teñidas con Hematoxilina y Eosina (H\&E).\newline

Originalmente, las muestras se clasifican en cinco categorías diagnósticas (NC, G3, G4 y G5). No obstante, para los fines de este estudio, se adoptó un esquema de clasificación binaria que distingue entre tejido benigno (NC) y tejido maligno (Gleason $\geq 6$). Esta dicotomización permite evaluar la capacidad del modelo para detectar malignidad en un escenario clínico realista, caracterizado por desbalance de clases y heterogeneidad morfológica.\newline

Cada WSI se asocia exclusivamente a una etiqueta global, sin anotaciones a nivel de parche, lo que define explícitamente un escenario de supervisión débil. Esta característica motiva la adopción de arquitecturas MIL, diseñadas para inferir patrones discriminativos a partir de información agregada.\newline

Dado que SICAPv2 mantiene una jerarquía estricta paciente--lámina, las particiones experimentales respetaron esta estructura para evitar fuga de información. De este modo, las métricas reportadas reflejan la capacidad real de generalización del sistema ante pacientes no observados durante el entrenamiento.

\begin{center}
    \captionof{table}{Configuración de hiperparámetros y entorno de entrenamiento.}
    \label{tab:hiperparametros}
    \vspace{2mm}
    \begin{tabular}{ll}
        \hline
        \textbf{Parámetro} & \textbf{Valor / Configuración} \\
        \hline
        Optimizador & Adam \\
        Tasa de aprendizaje & $1 \times 10^{-4}$ \\
        Función de pérdida & \texttt{BCEWithLogitsLoss} \\
        Tamaño de lote & 1 (nivel de bolsa) \\
        Número de épocas & 20 \\
        Entorno de cómputo & PyTorch / GPU (NVIDIA T4) \\
        \hline
    \end{tabular}
\end{center}

%-------------------------------------------------------------

\section{Extracción de características profundas}
Cada parche tisular fue procesado de manera independiente mediante una CNN para transformar la información visual de alta dimensionalidad en representaciones vectoriales compactas. Se empleó la arquitectura ResNet-50 preentrenada en ImageNet, truncando su capa totalmente conectada para obtener \textit{embeddings} de dimensión $2048$.\newline

Las representaciones resultantes se almacenaron de forma persistente, desacoplando explícitamente la etapa de extracción visual del proceso de entrenamiento MIL. Este diseño reduce significativamente el costo computacional y permite una experimentación eficiente con distintos esquemas de agregación.

%-------------------------------------------------------------
\section{Modelado mediante Aprendizaje de Instancias Múltiples}

Cada WSI se representa como una bolsa $\mathcal{B} = \{h_1, h_2, \ldots, h_n\}$ de vectores $h_i \in \mathbb{R}^{2048}$. El modelo MIL implementado utiliza un mecanismo de atención que asigna un coeficiente de importancia $\alpha_i$ a cada instancia, permitiendo una agregación ponderada de las regiones más relevantes desde el punto de vista diagnóstico.\newline

\begin{figure}[ht]
    \centering
    \includegraphics[width=0.6\textwidth]{images/DiagTeoricoArqMIL.png}
    \caption{Arquitectura del modelo MIL basado en atención implementado.}
    \label{fig:arquitectura_mil}
\end{figure}

Este marco de trabajo resulta particularmente adecuado para escenarios de supervisión débil, ya que evita que señales tumorales focales se diluyan dentro de grandes volúmenes de tejido benigno. Asimismo, los pesos de atención proporcionan una base sólida para la interpretabilidad clínica del modelo.

%-------------------------------------------------------------
\section{Configuración experimental y protocolo de entrenamiento}

Se implementó una estrategia de validación cruzada \textit{GroupKFold} con $k=5$ particiones, agrupando estrictamente los datos a nivel de paciente. Para cada iteración, el modelo se entrenó desde cero y se evaluó exclusivamente sobre WSIs pertenecientes a pacientes no observados durante el entrenamiento.\newline

El entrenamiento se realizó con un tamaño de lote unitario, coherente con la naturaleza variable del número de instancias por bolsa. Esta decisión evita el uso de técnicas de relleno artificial que podrían distorsionar la distribución real de los datos.

%-------------------------------------------------------------

\section{Métricas de evaluación del desempeño}
La evaluación del desempeño del modelo se realizó exclusivamente a nivel de WSI, en coherencia con el paradigma MIL y el contexto clínico del problema. Las métricas seleccionadas se alinean estrictamente con las definidas en el marco teórico:

\begin{itemize}
    \item \textbf{F1-score:} Métrica principal para evaluar el equilibrio entre precisión y sensibilidad en escenarios con desbalance de clases.
    \item \textbf{AUC-ROC:} Evalúa la capacidad discriminativa del modelo para ordenar correctamente WSIs benignas y malignas, independientemente del umbral de decisión.
    \item \textbf{Coeficiente de Cohen (Kappa):} Cuantifica el grado de concordancia entre las predicciones del modelo y las etiquetas clínicas reales, corrigiendo el acuerdo esperado por azar.
\end{itemize}

En todos los casos, las métricas se calcularon a partir de una única predicción agregada por WSI, obtenida mediante el mecanismo de atención MIL, sin realizar evaluaciones ni promedios a nivel de parche. Este esquema garantiza que los resultados reportados reflejen fielmente el desempeño clínico del sistema bajo condiciones de supervisión débil.\newline

Las métricas obtenidas a través de los cinco \textit{folds} se agregaron mediante estadísticos descriptivos (media y desviación estándar), proporcionando una estimación robusta del desempeño global del modelo. El análisis detallado de los resultados cuantitativos, las curvas de evaluación y los mapas de atención se presenta en el capítulo siguiente.



%%%%%%% MECANISMO DE ATENCIÓN %%%%%%%
\chapter{Análisis de Resultados y Mecanismo de Atención}
% -------------------------------------------------------------
% Capítulo: Análisis de Resultados y Atención (Consolidado)
% -------------------------------------------------------------

\section{Introducción y alcance del análisis}
Este capítulo presenta, analiza y discute los hallazgos derivados de la implementación del modelo de Aprendizaje de Instancias Múltiples (MIL) con mecanismo de atención. La exposición integra el control de calidad de los datos, la descripción técnica del flujo experimental, la evaluación cuantitativa mediante procedimientos estadísticos robustos y la validación cualitativa basada en la inspección de mapas de atención. \\

El objetivo central es validar no solo el rendimiento numérico del modelo en la discriminación entre tejido benigno y maligno (Gleason $\geq 3$), sino también su capacidad de explicabilidad mediante la identificación correcta de morfología patológica, un requisito indispensable para la traducción clínica.

\section{Análisis de Resultados Cuantitativos}

\subsection{Preprocesamiento y Control de Calidad de Datos}
La validez de cualquier aproximación computacional en histopatología depende de la integridad de los datos de entrada. Se implementó un protocolo riguroso de preprocesamiento y auditoría:

\subsection{Protocolo de extracción y normalización}
\begin{itemize}
    \item \textbf{Magnificación y tamaño de parche:} Extracción a una magnificación equivalente a 20x con dimensiones de $224\times224$ píxeles, coherente con la entrada del \textit{backbone} (ResNet-50) y ofreciendo un compromiso óptimo entre resolución morfológica y coste computacional.
    \item \textbf{Normalización de tinción:} Se aplicó normalización cromática tipo Macenko para reducir la variabilidad de color inter-centro y mejorar la generalización del extractor de características.
    \item \textbf{Trazabilidad:} El pipeline registra por parche la ruta WSI, coordenadas y métricas de calidad, permitiendo auditoría y replicación.
\end{itemize}

\subsection{Análisis de Fracción de Tejido (\textit{Tissue Fraction})}
Se calculó la \textit{tissue fraction} por parche aplicando un umbral de 0.05 para eliminar regiones vacías o ricas en fondo. 
\begin{itemize}
    \item \textbf{Volumen de datos:} El conjunto resultante incluyó \textbf{18,783 parches} provenientes del dataset SICAPv2.
    \item \textbf{Integridad biológica:} El análisis confirmó que el 100\% de las instancias superó el umbral de viabilidad ($\geq 0.05$). Como se observa en la Figura~\ref{fig:tissue_fraction_dist_1}, existe una alta densidad de tejido en la mayoría de las muestras (concentración predominante $>80\%$), garantizando que el modelo recibe señales histológicas reales.
\end{itemize}

\begin{figure}[ht!]
    \centering
    \includegraphics[width=0.6\textwidth]{images/tissue_fraction_dist.png} 
    \caption{Distribución de la fracción de tejido en el dataset consolidado. La mayoría de los parches presentan una ocupación superior al 80\%, asegurando contenido biológico suficiente.}
    \label{fig:tissue_fraction_dist_1}
\end{figure}

\section{Arquitectura y Protocolo Experimental}
El sistema se evaluó bajo un esquema de supervisión débil (etiquetas a nivel de WSI). Los componentes clave son:

\begin{itemize}
    \item \textbf{Extractor de características:} ResNet-50 preentrenada en ImageNet, con \textit{fine-tuning} en capas superiores para ajustar las representaciones a la morfología prostática.
    \item \textbf{Mecanismo de Atención:} Se implementó un bloque de atención tipo \textit{dot-product} modificado. Este mecanismo genera un peso de importancia $\alpha_i$ para cada parche $i$ a partir de su embedding $h_i$ (donde $h_i$ es el vector de características extraído por la CNN para el parche $i$). Matemáticamente:
        \begin{equation*}
            a_i = w^{\top} \tanh(V h_i^{\top}), \qquad \alpha_i = \frac{\exp(a_i)}{\sum_j \exp(a_j)}
        \end{equation*}
    La representación global de la bolsa (WSI) se obtiene mediante la suma ponderada de todos los embeddings usando los coeficientes $\alpha_i$:
        \begin{equation*}
            H = \sum_i \alpha_i h_i
        \end{equation*}
    \item \textbf{Estrategia de Validación:} Se utilizó \textbf{GroupKFold} estratificado por paciente para asegurar independencia clínica total entre las particiones de entrenamiento y validación.
\end{itemize}

\begin{figure}[ht!]
    \centering
    \includegraphics[width=0.8\textwidth]{images/DiagTeoricoArqMIL.png}
    \caption{Arquitectura del mecanismo de atención MIL propuesto. Se observa el flujo desde los embeddings de entrada ($h_i$) hasta la agregación pesada por los coeficientes $\alpha_i$ que determinan la contribución de cada parche a la representación global de la WSI.}
    \label{fig:arch_mil}
\end{figure}

\subsection{Procedimientos Estadísticos}
Para garantizar robustez, la evaluación trasciende el rendimiento puntual e incorpora métricas de incertidumbre:
\begin{itemize}
    \item \textbf{Bootstrap:} Estimación de intervalos de confianza (1,000 repeticiones) para ROC AUC y PR AUC.
    \item \textbf{Test de DeLong:} Comparación estadística de curvas ROC entre versiones del modelo.
    \item \textbf{Evaluación de calibración:} Análisis de curvas de fiabilidad y Brier score.
\end{itemize}

\subsection{Comparativa de Rendimiento: Versión A vs. Versión B}
Se evaluaron dos variantes del proceso de entrenamiento. Aunque ambas alcanzaron F1-scores promedio similares ($\approx 0.88$), el análisis detallado favorece a la \textbf{Versión A}:

\begin{itemize}
    \item \textbf{Curvas Precision-Recall (PR):} La Figura~\ref{fig:pr_curve_mil} muestra que la Versión A mantiene una precisión elevada hasta niveles de \textit{recall} medio-altos, logrando un PR AUC agregado de \textbf{0.917}. La Versión B muestra una caída prematura en precisión, lo que implica una mayor tasa de falsos positivos.
    \item \textbf{Estabilidad ROC:} La Figura~\ref{fig:roc_curve_mil} confirma que la Versión A presenta menor dispersión entre \textit{folds}, sugiriendo una mayor capacidad de generalización.
\end{itemize}

\begin{figure}[ht!]
    \centering
    \begin{minipage}{0.48\textwidth}
        \centering
        \includegraphics[width=\textwidth]{images/roc_curve_mil.png}
        \caption{Curvas ROC agregadas por folds (Versiones A y B).}
        \label{fig:roc_curve_mil}
    \end{minipage}\hfill
    \begin{minipage}{0.48\textwidth}
        \centering
        \includegraphics[width=\textwidth]{images/pr_curve_mil.png}
        \caption{Curvas PR agregadas por folds (Versiones A y B).}
        \label{fig:pr_curve_mil}
    \end{minipage}
\end{figure}

\subsection{Resumen Métrico por Fold (Versión A)}
La Tabla~\ref{tab:metrics_summaryA} consolida el rendimiento de la Versión A. El elevado PR AUC ($0.899 \pm 0.083$) es el hallazgo más relevante en un contexto clínico desbalanceado, indicando efectividad en la detección de la clase positiva (cáncer) minimizando falsos negativos.
\newpage
\begin{table}[ht!]
    \centering
    \caption{Métricas agregadas por fold (Media $\pm$ Desviación Estándar) para la Versión A.}
    \label{tab:metrics_summaryA}
    \begin{tabular}{lrr}
        \hline
        \textbf{Métrica} & \textbf{Media} & \textbf{Std} \\
        \hline
        ROC AUC & 0.795 & 0.070 \\
        PR AUC  & 0.899 & 0.083 \\
        F1 Score & 0.876 & 0.054 \\
        Precision (umbral 0.5) & 0.847 & 0.098 \\
        Recall (umbral 0.5)    & 0.916 & 0.034 \\
        \hline
    \end{tabular}
\end{table}

Además, la inspección de las curvas por fold individual (Figuras~\ref{fig:pr_per_fold} y \ref{fig:roc_per_fold}) revela una variabilidad controlada. Los comportamientos no idealizados en las curvas sugieren que el modelo aprende patrones genuinos y no está memorizando ruido (ausencia de sobreajuste).

\begin{figure}[ht!]
    \centering
    \includegraphics[width=0.8\textwidth]{images/pr_per_fold.png}
    \caption{Variabilidad de las curvas Precision-Recall por fold. La consistencia entre particiones valida la robustez de la arquitectura.}
    \label{fig:pr_per_fold}
\end{figure}

\begin{figure}[ht!]
    \centering
    \includegraphics[width=0.8\textwidth]{images/roc_per_fold.png}
    \caption{Variabilidad de las curvas ROC por fold.}
    \label{fig:roc_per_fold}
\end{figure}

\section{Validación Cualitativa y Explicabilidad}
Más allá de las métricas, es crítico verificar qué información utiliza el modelo para tomar decisiones.

\subsection{Distribución de Probabilidades y Confianza}
El histograma de probabilidades de la Versión A (Figura~\ref{fig:prob_dist_A}) muestra una marcada polarización hacia los extremos (0 y 1). Esta característica es altamente deseable ya que:
\begin{enumerate}
    \item Reduce la ambigüedad en la toma de decisiones (pocos casos en la "zona gris").
    \item Facilita la definición de umbrales operativos estables.
\end{enumerate}

\begin{figure}[ht!]
    \centering
    \includegraphics[width=0.6\textwidth]{images/prob_dist_A.png}
    \caption{Distribución de probabilidades estimadas por la Versión A. Se observa una clara separación entre clases con alta confianza.}
    \label{fig:prob_dist_A}
\end{figure}

\subsection{Análisis de Parches de Alta Atención (Top-k)}
El mecanismo de atención actúa como un selector de relevancia. Se extrajeron los parches con mayor coeficiente $\alpha_i$ en las WSIs clasificadas como malignas. La inspección visual del mosaico resultante (Figura~\ref{fig:top_patches}) demuestra una correspondencia directa entre los pesos de atención elevados y estructuras histopatológicas de interés:
\begin{itemize}
    \item Arquitectura glandular desorganizada (fusión glandular).
    \item Pleomorfismo nuclear y núcleos prominentes.
    \item Ausencia de atención en estroma sano o artefactos.
\end{itemize}
Este hallazgo valida cualitativamente que el modelo ha aprendido a localizar la enfermedad bajo supervisión débil.
\newpage
\begin{figure}[ht!]
    \centering
    \includegraphics[width=0.9\textwidth]{images/top_k_patches.png}
    \caption{Mosaico de parches \textit{top-k} (mayor atención). El modelo focaliza automáticamente regiones con características morfológicas de malignidad.}
    \label{fig:top_patches}
\end{figure}

\section{Discusión y Consideraciones Clínicas}

\subsection{Selección de Umbrales Operativos}
Aunque las métricas globales son positivas, la implementación clínica requiere definir un punto de corte. Dado el alto \textit{Recall} (0.916) al umbral 0.5, el modelo se perfila adecuadamente para tareas de cribado (\textit{screening}), donde la prioridad es no perder casos positivos. Se recomienda realizar un \textit{Decision Curve Analysis} (DCA) para ajustar este umbral según el beneficio neto esperado en diferentes escenarios de prevalencia.

\subsection{Robustez y Limitaciones}
\begin{itemize}
    \item \textbf{Supervisión Débil:} La falta de anotaciones a nivel de píxel o parche impide calcular una sensibilidad de localización exacta, aunque el análisis visual de \textit{Top-k} mitiga esta preocupación.
    \item \textbf{Validación Externa:} Los resultados, aunque robustos internamente (GroupKFold), provienen de una cohorte retrospectiva. Es imperativo testar el modelo en cohortes multicéntricas prospectivas para confirmar su generalización frente a variaciones técnicas de escaneo y tinción.
\end{itemize}

\section{Conclusión}
La Versión A del modelo MIL con atención presenta un equilibrio óptimo entre rendimiento cuantitativo (PR AUC $\approx 0.90$) y explicabilidad clínica. La validación cruzada del preprocesamiento, la estabilidad estadística entre folds y la coherencia morfológica de los mapas de atención consolidan esta arquitectura como una herramienta de apoyo diagnóstico prometedora. Los siguientes pasos deben orientarse hacia la validación externa y estudios de interacción patólogo-IA (\textit{reader studies}) para medir el impacto real en el flujo de trabajo clínico.

%%%%%%% DIFICULTADES y CONCLUSIONES %%%%%%%%%%%%%
\chapter{Dificultades}
La implementación del proyecto de grado se enfrentó a desafíos significativos debido a
\chapter{Conclusiones y Trabajo Futuro}
\section{Conclusiones Preliminares}

Aunque este proyecto de grado se encuentra en una fase de desarrollo, se han logrado avances significativos en la identificación de variables clave que influyen en la implementación de la Creación de Valor Compartido (CVC) en las pequeñas y medianas empresas (Pymes) de Barranquilla. La revisión de la literatura y el análisis del contexto local han permitido establecer una base teórica sólida para el diseño de un sistema de información adaptado a las necesidades de estas organizaciones.\\

El desarrollo del modelo predictivo multivariante ha proporcionado una comprensión más profunda de las relaciones entre las variables identificadas, lo que contribuirá a la formulación de estrategias efectivas para la adopción de prácticas sostenibles en las Pymes. Estos avances preliminares sientan las bases para futuras investigaciones y aplicaciones prácticas en el ámbito empresarial local.

\section{Trabajos Futuros}

A medida que el proyecto avance, se contemplan las siguientes líneas de trabajo para fortalecer y ampliar los resultados obtenidos:

\begin{itemize}
    \item \textbf{Validación Empírica del Modelo}: Aplicar el modelo predictivo desarrollado a una muestra representativa de Pymes en Barranquilla para evaluar su precisión y utilidad en contextos reales.
    \item \textbf{Desarrollo de Herramientas Digitales}: Diseñar e implementar un sistema de información basado en los hallazgos del estudio, que facilite la toma de decisiones y la adopción de prácticas de CVC en las Pymes.
    \item \textbf{Capacitación y Sensibilización}: Elaborar programas de formación dirigidos a empresarios y empleados de Pymes para promover la comprensión y aplicación de estrategias de CVC.
    \item \textbf{Ampliación del Estudio}: Extender la investigación a otras regiones y sectores económicos para evaluar la generalizabilidad del modelo y adaptar las estrategias a diferentes contextos empresariales.
\end{itemize}

Estos trabajos futuros contribuirán a consolidar los esfuerzos hacia una economía más sostenible y socialmente responsable en la región.


%%%%%%% CRONOGRAMA %%%%%%%%%%%%%
%\chapter{Cronograma}
%\subsection{Cronograma}
El tiempo total del proyecto se llevará a cabo en 24 semanas las cuales se distribuyen de la siguiente manera:

\begin{figure}[h!]
    \centering
    \includegraphics[width=1.0\textwidth]{images/CRONOGRAMA_PROYECTO3.png}
    \caption{Cronograma del proyecto (Elaboración propia)}
    \label{fig:cronograma}
\end{figure}



%%%%%% ANEXOS %%%%%%%%%%%%%
\chapter{Anexos}
\begin{itemize}
     \item \href{https://drive.google.com/file/d/1L7VoDJk_7w4tyqen5Lit_a5-qwMPtpGx/view?usp=sharing}{Cronograma de actividades}
     \item \href{https://drive.google.com/file/d/114DsgGKj7mDv4HfXlW0STJWhqu4Khwid/view?usp=sharing}{Cuaderno de Implementación del Modelo Attention-MIL y Flujo de Ingeniería de Datos.}
     \item \href{https://github.com/JuanjoRestrepo/Trabajo-de-Grado-WSI-MIL}{Repositorio de Github del Proyecto de Investigación}
    % \item \href{https://github.com/JuanjoRestrepo/TESIS-2023/tree/main/RoboDK%20Simulacion}{Simulación RoboDK}
    % \item \href{https://github.com/JuanjoRestrepo/TESIS-2023/tree/main/RoboDK%20Simulacion/Scripts%20ROBODK}{Software de Control}
    % \item \href{https://github.com/JuanjoRestrepo/TESIS-2023/blob/main/GUI%20Celda%20Manufactura/Graph.py}{Script de consultas a Base de Datos}
    % \item \href{https://github.com/JuanjoRestrepo/TESIS-2023/blob/main/GUI%20Celda%20Manufactura/Dashboard.py}{Script de comunicación a Dashboard}
    % \item \href{https://drive.google.com/drive/folders/1FiFPI-KyEd_kaHG5zytG-qa89Rc2Fpt7?usp=sharing}{Prueba Crear Orden} 
    % \item \href{https://drive.google.com/drive/folders/1FvAPYssqlqL0Cqc-dVg2ZfjFK7EL4WSv?usp=sharing}{Prueba Modificar Orden}
    % \item \href{https://drive.google.com/drive/folders/1FtKB4AIjtJiD6wF7VAdpiTrXkQoNfQ_3?usp=sharing}{Prueba Eliminar Orden}
    % \item \href{https://drive.google.com/drive/folders/1GB3DZiQTv8dBRW6YTMn4ey4d3KmOQ6dR?usp=sharing}{Prueba Ver Órdenes}
    % \item \href{https://drive.google.com/drive/folders/1GBdRwiYlig3eHsq7P3kORDGBVyIgWRSq?usp=sharing}{Prueba Ejecutar Celda}
    % \item \href{https://drive.google.com/drive/folders/1FwzjEMfWkMELAdxN0yd_n16XujeNhsHe?usp=sharing}{Prueba Re-establecer Almacén}
    % \item \href{https://lookerstudio.google.com/reporting/0919ab13-5849-4f13-b416-253a44027fbd}{Dashboard}
    % \item \href{https://github.com}{Repositorio de Github}
\end{itemize}


%%%%%%%%%%%%%%%%%%%%%%%%%%%%%
% BIBLIOGRAFIA
%%%%%%%%%%%%%%%%%%%%%%%%%%%%%
\bibliographystyle{IEEEtran}
\bibliography{biblio}

%%%%%%%%%%%%%%%%
% GENERAL INDEX
%%%%%%%%%%%%%%%%
\printindex

\end{document}
\endinput

%%
%% End of file `main.tex'.



