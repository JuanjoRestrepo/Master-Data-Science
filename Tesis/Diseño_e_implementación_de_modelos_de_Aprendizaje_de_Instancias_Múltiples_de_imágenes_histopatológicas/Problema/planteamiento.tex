El cáncer de próstata constituye una de las principales causas de morbilidad y mortalidad en la población masculina a nivel mundial. Según estimaciones recientes de la Organización Mundial de la Salud (OMS) y del Global Cancer Observatory (GCO), se reportan anualmente más de 1.4 millones de casos nuevos, con una carga especialmente elevada en países de ingresos medios y bajos debido a barreras en el acceso al diagnóstico temprano y al tratamiento oportuno \cite{SungH_2021}. En el contexto colombiano, esta enfermedad figura entre las primeras causas de muerte por cáncer en hombres, lo que subraya la necesidad de fortalecer estrategias de detección precoz y estratificación de riesgo que apoyen la toma de decisiones clínicas \cite{MinSalud_2025}.\newline

El diagnóstico histopatológico mediante la escala de Gleason constituye el pilar de la estratificación pronóstica; sin embargo, su aplicación está sujeta a una variabilidad interobservador significativa. Esta subjetividad es especialmente crítica en la discriminación entre patrones limítrofes (por ejemplo, Gleason 3 frente a Gleason 4) y en la identificación de subtipos histológicos asociados a peor pronóstico, como el patrón \emph{cribriforme}. Dicho patrón se caracteriza por formaciones glandulares con arquitectura tipo criba y se asocia a mayor agresividad tumoral y riesgo de recurrencia, por lo que su correcta detección tiene implicaciones terapéuticas relevantes \cite{SILVARODRIGUEZ2020105637} \cite{Brett_2016}.\newline

La digitalización de muestras histopatológicas y la generación de \textit{Whole Slide Images} (WSI) han habilitado el análisis computacional del tejido completo, pero introducen retos técnicos sustanciales. Entre ellos se incluyen el tamaño gigapíxel de las imágenes, la heterogeneidad morfológica intratumoral, las variaciones de tinción entre centros y la presencia de artefactos derivados de la preparación y el escaneo de las muestras \cite{Tellez_2018}. Adicionalmente, la disponibilidad de conjuntos de datos con anotaciones detalladas a nivel de parche (\textit{patch-level}) es limitada, debido al alto costo y tiempo requeridos para su generación por expertos.\newline

Es importante aclarar la diferencia entre los niveles de anotación que condicionan la metodología de aprendizaje. Una etiqueta a nivel de lámina/muestra (\textit{slide-level}) asigna una única etiqueta/clase a toda la WSI, mientras que una etiqueta a nivel de instancia (\textit{patch-level}) identifica la clase de cada parche individual. La ausencia generalizada de anotaciones \textit{patch-level} ha motivado el uso de enfoques de aprendizaje débil, particularmente el paradigma de \textit{Multiple Instance Learning}, que permite entrenar modelos utilizando únicamente etiquetas globales a nivel de WSI.\newline

Desde la perspectiva de la ciencia de datos, la combinación de imágenes de gran escala, supervisión débil y heterogeneidad interinstitucional genera un conjunto de desafíos metodológicos bien definidos. Estos incluyen la necesidad de estrategias eficientes de preprocesamiento y representación, el manejo del desbalance de clases, la evaluación robusta del desempeño y la mitigación de \textit{domain shifts} derivados de diferencias en protocolos de tinción y adquisición \cite{campanella2019clinical}\cite{Tellez_2018}. Asimismo, la escasez de subtipos relevantes desde el punto de vista clínico plantea retos adicionales para el aprendizaje y la generalización de los modelos.\newline

En este contexto, el problema central que aborda este proyecto consiste en diseñar y evaluar un pipeline de aprendizaje débil capaz de clasificar imágenes histopatológicas prostáticas a partir de etiquetas \textit{slide-level}, preservando información diagnóstica relevante y garantizando una evaluación experimental reproducible. Particularmente, se investiga el uso de variantes de MIL basadas en mecanismos de atención, con el fin de analizar no solo su desempeño cuantitativo, sino también su estabilidad y potencial interpretativo.\newline

Finalmente, el proyecto adopta prácticas esenciales de reproducibilidad, tales como particiones estratificadas por paciente, control de semillas aleatorias y versionado del conjunto de parches, que permiten realizar comparaciones internas consistentes entre distintas configuraciones del modelo. Evaluaciones más extensas, como validaciones interinstitucionales o análisis cualitativos con múltiples patólogos, se reconocen como relevantes, pero se consideran fuera del alcance de este proyecto y se proponen como líneas de trabajo futuro.

