Small and medium-sized enterprises (SMEs) constitute the backbone of the Colombian economy, playing a crucial role in job creation and business dynamism. However, these organizations face several significant challenges in integrating sustainable practices into their operations. These include limited financial capacity, limited access to clean technologies, limited training in environmental management, and a lack of government incentives. These factors, combined with the constant pressure to remain competitive in increasingly demanding markets, lead many SMEs to prioritize their operational and economic growth strategies, neglecting environmental considerations that could significantly contribute to shared value creation (SVC).\\

In response to this situation, the project aims to develop a multivariate model to predict shared value creation and its relationship with environmental sustainability in Colombian SMEs. Inspired by the integration of key variables that reflect both the internal situation of each company (such as operational capacity, resource availability, and level of technical knowledge) and external environmental factors (such as market dynamics, government policies, and sustainability trends), this model is proposed as a predictive and practical tool for strategic decision-making. The idea is to identify, through statistical analysis and the use of advanced modeling techniques, which variables have the greatest influence on the successful implementation of sustainable practices and, in this way, generate a framework that guides SMEs toward a comprehensive business transformation.\\

The proposed approach not only seeks to improve the competitiveness of SMEs but also to promote a reorientation of their processes toward a more resilient and responsible economy. With this model, it is expected that companies will not only optimize their economic performance but also generate social and environmental benefits, strengthening the concept of shared value creation. Thus, the project contributes to a better understanding of the challenges and opportunities in the integration of sustainable practices, offering an innovative solution that, in addition to facilitating decision-making, promotes the adoption of strategies that benefit both the company and society at large.\\

In summary, this study is positioned as a comprehensive response to the current limitations in the implementation of sustainability measures in Colombian SMEs, by combining predictive analysis with the evaluation of contextual variables. The application of this multivariate model is expected not only to generate improvements in the operational efficiency and competitiveness of companies, but also to contribute decisively to the country's sustainable development by facilitating an alignment between economic objectives and environmental and social needs.\\

\textbf{Keywords:} Creating Shared Value, Multivariate Models, SMEs, Environmental Sustainability.

% %Graph database, Manufacturing cells, Neo4j, Cypher, Manufacturing execution systems.
