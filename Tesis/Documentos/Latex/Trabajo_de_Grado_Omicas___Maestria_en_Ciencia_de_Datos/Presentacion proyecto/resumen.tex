Las pequeñas y medianas empresas (Pymes) constituyen la columna vertebral de la economía colombiana, desempeñando un papel crucial en la generación de empleo y en el dinamismo del tejido empresarial. Sin embargo, estas organizaciones enfrentan una serie de desafíos importantes para integrar prácticas sostenibles en sus operaciones. Entre ellos se destacan la limitada capacidad financiera, el difícil acceso a tecnologías limpias, la escasa formación en gestión ambiental y la falta de incentivos gubernamentales. Estos factores, combinados con la presión constante por mantenerse competitivas en mercados cada vez más exigentes, hacen que muchas Pymes prioricen sus estrategias operativas y de crecimiento económico, dejando de lado consideraciones ambientales que podrían contribuir significativamente a la creación de valor compartido (CVC).\\

En respuesta a este panorama, el proyecto se propone desarrollar un modelo multivariante que permita predecir la creación de valor compartido y su relación con la sostenibilidad ambiental en las Pymes colombianas. Inspirándose en la integración de variables clave que reflejen tanto la situación interna de cada empresa (como la capacidad operativa, disponibilidad de recursos y nivel de conocimiento técnico) como factores externos del entorno (como las dinámicas del mercado, políticas gubernamentales y tendencias en sostenibilidad), este modelo se plantea como una herramienta predictiva y práctica para la toma de decisiones estratégicas. La idea es que, mediante el análisis estadístico y el uso de técnicas avanzadas de modelado, se pueda identificar qué variables tienen mayor influencia en el éxito de la implementación de prácticas sostenibles y, de esta forma, generar un marco de referencia que oriente a las Pymes hacia una transformación empresarial integral.\\

El enfoque propuesto no solo busca mejorar la competitividad de las Pymes, sino también impulsar una reorientación de sus procesos hacia una economía más resiliente y responsable. Con este modelo, se espera que las empresas no solo puedan optimizar su rendimiento económico, sino que también generen beneficios sociales y ambientales, fortaleciendo el concepto de creación de valor compartido. Así, el proyecto contribuye a una mayor comprensión de los retos y oportunidades en la integración de prácticas sostenibles, ofreciendo una solución innovadora que, además de facilitar la toma de decisiones, promueve la adopción de estrategias que benefician tanto a la empresa como a la sociedad en general.\\

En síntesis, este estudio se posiciona como una respuesta integral a las limitaciones actuales en la implementación de medidas de sostenibilidad en las Pymes colombianas, al combinar el análisis predictivo con la evaluación de variables contextuales. Se espera que la aplicación de este modelo multivariante no solo genere mejoras en la eficiencia operativa y competitividad de las empresas, sino que también contribuya de manera decisiva al desarrollo sostenible del país, al facilitar un alineamiento entre los objetivos económicos y las necesidades ambientales y sociales.\\

\textbf{Palabras Claves:} Creación de Valor Compartido, Modelos Multivariantes, Pymes, Sostenibilidad Ambiental.

% %Base de datos de grafos, Celdas de manufactura, Neo4j, Cypher, Sistemas de ejecución de manufactura.