% Al analizar las características y necesidades de las Pymes en Barranquilla, se han establecido los siguientes requisitos para el diseño de un sistema que facilite la implementación efectiva de la Creación de Valor Compartido:

% \begin{enumerate}
%     \item \textbf{Adaptabilidad y Escalabilidad del Sistema}
%     \begin{enumerate}
%         \item El sistema debe ser adaptable a diferentes sectores y tamaños de Pymes, permitiendo su escalabilidad y personalización según las necesidades específicas de cada empresa.
%         \item Debe permitir la integración con herramientas existentes y futuras, facilitando la evolución tecnológica y organizacional de las Pymes.
%     \end{enumerate}
    
%     \item \textbf{Gestión de Datos para la Sostenibilidad}
%     \begin{enumerate}
%         \item El sistema debe contar con una base de datos robusta que permita el almacenamiento y análisis de indicadores clave relacionados con la sostenibilidad ambiental, social y económica.
%         \item Debe facilitar la recopilación de datos sobre consumo de recursos, emisiones, gestión de residuos y otros aspectos ambientales relevantes.
%     \end{enumerate}
    
%     \item \textbf{Interfaz de Usuario Intuitiva}
%     \begin{enumerate}
%         \item El sistema debe ofrecer una interfaz de usuario amigable que permita a los empleados de las Pymes ingresar y consultar información de manera eficiente.
%         \item Debe incluir módulos específicos para la gestión de proyectos de CVC, seguimiento de indicadores de sostenibilidad y generación de informes.
%     \end{enumerate}
    
%     \item \textbf{Soporte para la Toma de Decisiones Estratégicas}
%     \begin{enumerate}
%         \item El sistema debe proporcionar herramientas analíticas que faciliten la identificación de oportunidades para la creación de valor compartido y la mejora continua en sostenibilidad.
%         \item Debe permitir la simulación de escenarios y el análisis de impacto de diferentes estrategias empresariales en términos de sostenibilidad.
%     \end{enumerate}
    
%     \item \textbf{Cumplimiento Normativo y Reporte de Información}
%     \begin{enumerate}
%         \item El sistema debe estar alineado con las normativas locales e internacionales relacionadas con la sostenibilidad y la responsabilidad social empresarial.
%         \item Debe facilitar la generación de informes requeridos por entidades regulatorias y partes interesadas, asegurando la transparencia y rendición de cuentas.
%     \end{enumerate}
    
%     \item \textbf{Capacitación y Soporte Técnico}
%     \begin{enumerate}
%         \item El sistema debe incluir recursos de capacitación para los usuarios, promoviendo una cultura organizacional orientada a la sostenibilidad y la innovación.
%         \item Debe contar con soporte técnico accesible para resolver problemas y garantizar el funcionamiento continuo del sistema.
%     \end{enumerate}
% \end{enumerate}