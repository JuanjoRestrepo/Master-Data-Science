\section{Objetivo General}
Transformar el conocimiento tácito y los datos heterogéneos generados por las investigaciones del Instituto iÓMICAS en información estructurada y procesable, mediante técnicas avanzadas de aprendizaje automático, con el propósito de identificar patrones, sistematizar experiencias y fortalecer la toma de decisiones para el desarrollo científico y tecnológico.

\section{Objetivos Específicos}

\begin{enumerate}
  \item \textbf{Captura y sistematización de conocimiento:}  
    Recopilar de manera sistemática el conocimiento tácito y los datos heterogéneos (cualitativos y cuantitativos) generados por especialistas, investigadores y equipos de medición del Instituto iÓMICAS, a través de entrevistas semiestructuradas, talleres y la extracción de metadatos de archivos experimentales existentes.

  \item \textbf{Estandarización documental y metadatos:}  
    Diseñar un esquema documental y de metadatos flexible, basado en ontologías y estándares abiertos (p.ej., OBI, MIBBI), que integre datos experimentales, metodológicos y resultados analíticos, y permita el cumplimiento de los principios FAIR (Findable, Accessible, Interoperable, Reusable).

  \item \textbf{Desarrollo de pipelines de IA:}  
    Implementar pipelines automatizados de procesamiento y preprocesamiento de datos basados en aprendizaje automático (normalización, detección de patrones, reducción de dimensionalidad, anotación), que extraigan correlaciones y tendencias relevantes para apoyar la generación de hipótesis científicas.

  \item \textbf{Validación y ajuste con expertos:}  
    Confrontar los modelos analíticos y la estructura documental con investigadores y usuarios del semillero, mediante sesiones de validación y métricas de desempeño (precisión, interpretabilidad), para garantizar la aplicabilidad práctica y coherencia con los procesos reales de investigación.

  \item \textbf{Retroalimentación continua y sostenibilidad:}  
    Establecer mecanismos iterativos de seguimiento, evaluación y actualización dinámica del sistema (versionado de metadatos, registros de cambios y nuevos pipelines), a fin de incorporar nuevas líneas de investigación, asegurar la adaptabilidad tecnológica y mantener la sostenibilidad del conocimiento a largo plazo.
\end{enumerate}