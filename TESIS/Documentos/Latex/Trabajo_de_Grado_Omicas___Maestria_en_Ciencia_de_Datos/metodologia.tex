\section{Metodología}

El enfoque metodológico de este trabajo se estructura en cinco fases principales, orientadas a transformar el conocimiento tácito y los datos heterogéneos del Instituto iÓmicas en información estructurada y automatizada, con apoyo de técnicas de aprendizaje automático.

\subsection{Fase 1: Recolección y sistematización de datos}
\begin{itemize}
  \item Se realizarán entrevistas semiestructuradas y talleres con especialistas e investigadores para capturar conocimiento experto no documentado.
  \item Se recopilarán datos experimentales, metodológicos y analíticos generados en proyectos anteriores (hojas de cálculo, protocolos, archivos brutos y procesados).
  \item Todo material se indexará y se registrarán metadatos de fuente, fecha, autor y contexto experimental.
\end{itemize}

\subsection{Fase 2: Diseño de estructura documental}
\begin{itemize}
  \item Se propondrá un esquema flexible que combine ontologías (como OBI u OBO Foundry) y estándares mínimos (MIBBI, MSI) para describir experimentos ómicos \cite{OBI,MSI_framework}.
  \item Se integrará una base de conocimiento semántica con URIs, definiciones, control de versiones y referencias cruzadas a vocabularios reconocidos \cite{OBI} \cite{MSI_framework}.
  \item Las tablas y archivos se reorganizarán respetando principios FAIR: identificadores persistentes, metadatos ricos, protocolos accesibles y formatos abiertos \cite{BeFAIR} \cite{DatosFAIR}.
\end{itemize}

\subsection{Fase 3: Procesamiento y análisis automatizado}
\begin{itemize}
  \item Se implementará un pipeline inspirado en normas de metabolómica FAIR, con control de calidad, preprocesamiento y trazabilidad computacional \cite{QualityMetabolomics} \cite{ImplementationFAIR}.
  \item El pipeline incluirá pasos de normalización, detección de patrones estadísticos, anotación y predicción automatizada con machine learning.
  \item Se documentarán versiones del software, parámetros y entorno de ejecución (e.g. Docker, nf-core) para asegurar reproducibilidad \cite{FAIR_multomics}.
\end{itemize}

\subsection{Fase 4: Validación con expertos}
\begin{itemize}
  \item Se presentarán prototipos de documentación y resultados a investigadores del semillero para retroalimentación.
  \item Se verificarán la coherencia, usabilidad y aplicación práctica en casos reales.
  \item Se medirán indicadores de usabilidad, completitud semántica y satisfacción usuario.
\end{itemize}

\subsection{Fase 5: Mecanismos de retroalimentación y actualización}
\begin{itemize}
  \item Se definirán procesos para incorporar nuevo conocimiento, datos o resultados al sistema de forma continua.
  \item Se diseñará un entorno de gestión versionado que mantenga trazabilidad histórica y facilite adaptaciones a futuras líneas ómicas.
  \item Se aplicarán actualizaciones automáticas sobre vocabularios, ontologías y estándares, garantizando la conformidad con cambios institucionales o tecnológicos.
\end{itemize}

\subsection{Aspectos técnicos adicionales}
\begin{itemize}
  \item La ontología documental se desarrollará con OWL y herramientas como ROBOT o el Ontology Development Kit \cite{ODK}.
  \item El motor semántico y generación de metadatos usará RDF/OWL para interoperabilidad \cite{BestPracticesOntologies}.
  \item El pipeline de análisis se implementará en un entorno reproducible (Docker/Singularity, CI/CD) y se depositará en repositorios como Zenodo o WorkflowHub para obtener DOI \cite{FAIR_multomics}.
  \item Se integrarán repositorios públicos (MetaboLights, Metabolomics Workbench) para cumplir con estándares FAIR y conectar con datos referenciales \cite{MetaboLights} \cite{MetabolomicsWorkbench}.
\end{itemize}
