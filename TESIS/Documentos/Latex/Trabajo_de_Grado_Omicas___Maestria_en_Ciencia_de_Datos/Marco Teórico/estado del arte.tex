% Para la realización de este trabajo se consultaron diferentes investigaciones enfocadas en el análisis de sostenibilidad y creación de valor compartido en pequeñas y medianas empresas, mediante el uso de modelos predictivos y enfoques multivariantes. Estos trabajos se resumen a continuación:

% % ----------------------------------------------------------------------------------------
% \subsection{Using structural equation modeling to test environmental performance in small and medium-sized manufacturers: can SEM help SMEs?}
% En este artículo, los autores exploran la utilidad del modelado de ecuaciones estructurales (SEM) como metodología para validar un modelo de desempeño ambiental dirigido a pequeñas y medianas empresas manufactureras (PYMEs), particularmente en los sectores de plásticos y metales fabricados. El modelo propuesto se fundamenta en los Criterios Baldrige de Excelencia, adaptados para enfocar la gestión ambiental dentro de las PYMEs.\\

% El estudio se basó en una muestra de 458 empresas, empleó el Modelado de Ecuaciones Estructurales (SEM) mediante la herramienta LISREL para examinar las interconexiones entre variables latentes clave: liderazgo, planificación, recursos humanos, gestión ambiental, análisis de información y resultados ambientales. Si bien el modelo global presentó un ajuste razonable, algunas relaciones con los resultados ambientales no fueron estadísticamente significativas, lo que sugiere que las PYMEs enfrentan dificultades para comprender y medir eficazmente este concepto. Además, se enfatiza la urgencia de refinar la definición de los resultados ambientales y de adaptar las herramientas y modelos preexistentes, diseñados principalmente para grandes corporaciones, a las condiciones operativas específicas de las PYMEs \cite{hussey2007using}.


% % ----------------------------------------------------------------------------------------
% \subsection{Using structural equation modeling to test environmental performance
% in small and medium-sized manufacturers: can SEM help SMEs}
% En este artículo se analiza la aplicación del modelado de ecuaciones estructurales (SEM) para la validación de modelos de desempeño ambiental en pequeñas y medianas empresas (PYMEs). Los autores describen cómo la utilización de SEM permite establecer relaciones causales entre variables latentes que representan componentes esenciales en la gestión ambiental, tales como el liderazgo, la planificación, la implicación de recursos humanos y la implementación de prácticas de gestión ambiental.\\

% El estudio, basado en la adaptación de criterios internacionales (como los Criterios Baldrige) al contexto de las PYMEs, demuestra que la metodología proporciona un ajuste razonable del modelo global. Sin embargo, se destaca la necesidad de precisar la definición de “resultados ambientales”, ya que algunos caminos del modelo no resultaron estadísticamente significativos. Este trabajo es relevante para el presente estudio, ya que respalda la viabilidad de enfoques multivariantes para integrar aspectos de sostenibilidad ambiental y creación de valor compartido en las PYMEs \cite{hussey2007ecuaciones}.\\

% Estos hallazgos son pertinentes para nuestro trabado, dado que analiza el contexto de las Pymes implementando SEM y la relación entre variables latentes, ofreciendo resultados técnicos sobre la pertinencia de algunas variables. Dichos hallazgos resultan importantes como insumo para identificar y descartar variables del modelo de estimación.

% % ----------------------------------------------------------------------------------------
% \subsection{The impact of creating shared value strategy on corporate sustainable development: From resources perspective}
% En este artículo, Li et al. (2023) examinan el impacto de la estrategia de creación de valor compartido en el desarrollo sostenible corporativo, desde una perspectiva centrada en los recursos. El estudio se enfoca en cómo la integración y gestión de recursos en pequeñas y medianas empresas (PYMEs) puede potenciar tanto la competitividad como la sostenibilidad ambiental y social. A través de un análisis empírico y el uso de modelos multivariantes, los autores identifican factores críticos que facilitan la implementación exitosa de estrategias de valor compartido, destacando la importancia de alinear los objetivos económicos con los ambientales. Este enfoque permite a las PYMEs generar ventajas competitivas sostenibles, mejorando su desempeño global y promoviendo prácticas que contribuyen al desarrollo sostenible. La investigación aporta un marco conceptual y metodológico relevante para el presente trabajo, ya que respalda la integración de modelos multivariantes para predecir la creación de valor compartido y su relación con la sostenibilidad ambiental en el contexto de las PYMEs \cite{li2023impact}.

% % ----------------------------------------------------------------------------------------
% \subsection{Strategic Sense-Making and Value Creation in SMEs}
% En este artículo se explora el rol del sense-making estratégico en la creación de valor en las PYMEs. Los autores analizan cómo los procesos cognitivos y la interpretación de la información por parte de los directivos permiten identificar y aprovechar oportunidades para generar valor compartido. En este sentido, el estudio destaca que la capacidad de sentido estratégico facilita la alineación de los recursos internos con las demandas del entorno, lo que resulta fundamental para impulsar innovaciones y mejorar la competitividad. Mediante un enfoque empírico, se examinan casos y se aplican métodos cuantitativos para evidenciar que un adecuado proceso de sense-making contribuye a fortalecer la capacidad de las PYMEs para generar valor tanto interno como en su relación con el mercado. Estos hallazgos son relevantes para el presente trabajo, ya que aportan elementos teóricos y prácticos que pueden integrarse en modelos multivariantes orientados a predecir la creación de valor compartido y su impacto en la sostenibilidad ambiental \cite{SenseMakingValueCreation}.

% % ----------------------------------------------------------------------------------------
% \subsection{Determinants of Environmental, Financial, and Social Sustainability}
% En este artículo se analizan de manera integral los determinantes que influyen en la sostenibilidad en tres dimensiones fundamentales: ambiental, financiera y social. Los autores adoptan un enfoque multivariante para identificar cómo factores internos —como la disponibilidad y la gestión de recursos, las capacidades organizacionales y las prácticas de innovación—, junto con elementos externos del entorno competitivo y regulatorio, inciden en el desempeño sostenible de las organizaciones.\\

% \noindent El estudio proporciona evidencia empírica de que la alineación estratégica y la integración de políticas de sostenibilidad en el modelo de negocio son cruciales para mejorar el desempeño en las tres áreas analizadas. Aunque la investigación no se centra exclusivamente en PYMEs, los hallazgos ofrecen importantes lecciones sobre cómo las pequeñas y medianas empresas pueden optimizar la creación de valor compartido mediante la adopción de estrategias que combinen la eficiencia ambiental con resultados financieros y beneficios sociales. Esto respalda la premisa del presente trabajo, que propone el desarrollo de modelos multivariantes para predecir la creación de valor compartido y su relación con la sostenibilidad ambiental en PYMEs \cite{DeterminantsSustainability}.

