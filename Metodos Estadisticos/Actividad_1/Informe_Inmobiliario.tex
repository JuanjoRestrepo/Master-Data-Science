% Options for packages loaded elsewhere
\PassOptionsToPackage{unicode}{hyperref}
\PassOptionsToPackage{hyphens}{url}
%
\documentclass[
]{article}
\usepackage{amsmath,amssymb}
\usepackage{iftex}
\ifPDFTeX
  \usepackage[T1]{fontenc}
  \usepackage[utf8]{inputenc}
  \usepackage{textcomp} % provide euro and other symbols
\else % if luatex or xetex
  \usepackage{unicode-math} % this also loads fontspec
  \defaultfontfeatures{Scale=MatchLowercase}
  \defaultfontfeatures[\rmfamily]{Ligatures=TeX,Scale=1}
\fi
\usepackage{lmodern}
\ifPDFTeX\else
  % xetex/luatex font selection
\fi
% Use upquote if available, for straight quotes in verbatim environments
\IfFileExists{upquote.sty}{\usepackage{upquote}}{}
\IfFileExists{microtype.sty}{% use microtype if available
  \usepackage[]{microtype}
  \UseMicrotypeSet[protrusion]{basicmath} % disable protrusion for tt fonts
}{}
\makeatletter
\@ifundefined{KOMAClassName}{% if non-KOMA class
  \IfFileExists{parskip.sty}{%
    \usepackage{parskip}
  }{% else
    \setlength{\parindent}{0pt}
    \setlength{\parskip}{6pt plus 2pt minus 1pt}}
}{% if KOMA class
  \KOMAoptions{parskip=half}}
\makeatother
\usepackage{xcolor}
\usepackage[margin=1in]{geometry}
\usepackage{graphicx}
\makeatletter
\def\maxwidth{\ifdim\Gin@nat@width>\linewidth\linewidth\else\Gin@nat@width\fi}
\def\maxheight{\ifdim\Gin@nat@height>\textheight\textheight\else\Gin@nat@height\fi}
\makeatother
% Scale images if necessary, so that they will not overflow the page
% margins by default, and it is still possible to overwrite the defaults
% using explicit options in \includegraphics[width, height, ...]{}
\setkeys{Gin}{width=\maxwidth,height=\maxheight,keepaspectratio}
% Set default figure placement to htbp
\makeatletter
\def\fps@figure{htbp}
\makeatother
\setlength{\emergencystretch}{3em} % prevent overfull lines
\providecommand{\tightlist}{%
  \setlength{\itemsep}{0pt}\setlength{\parskip}{0pt}}
\setcounter{secnumdepth}{-\maxdimen} % remove section numbering
\ifLuaTeX
  \usepackage{selnolig}  % disable illegal ligatures
\fi
\usepackage{bookmark}
\IfFileExists{xurl.sty}{\usepackage{xurl}}{} % add URL line breaks if available
\urlstyle{same}
\hypersetup{
  pdftitle={Informe de Análisis del Mercado Inmobiliario en Cali},
  pdfauthor={Juan José Restrepo Rosero},
  hidelinks,
  pdfcreator={LaTeX via pandoc}}

\title{Informe de Análisis del Mercado Inmobiliario en Cali}
\author{Juan José Restrepo Rosero}
\date{31 July 2024}

\begin{document}
\maketitle

\section{Introducción}\label{introducciuxf3n}

En este informe se presenta un análisis descriptivo del mercado
inmobiliario en la ciudad de Cali, basado en los datos recopilados por
la empresa B\&C.

\section{Objetivos}\label{objetivos}

\begin{enumerate}
\def\labelenumi{\arabic{enumi}.}
\tightlist
\item
  Analizar el precio de las viviendas en diferentes zonas de Cali.
\item
  Identificar el tipo de viviendas más ofertadas en Cali.
\item
  Determinar las características más relevantes de la oferta de vivienda
  en Cali.
\end{enumerate}

\section{Métodos}\label{muxe9todos}

Se utilizaron técnicas de análisis descriptivo para identificar
tendencias y patrones en los datos.

\section{Resultados}\label{resultados}

\subsection{Precio de las viviendas en diferentes zonas de
Cali}\label{precio-de-las-viviendas-en-diferentes-zonas-de-cali}

\begin{verbatim}
{r}
# Cargar librerías
library(dplyr)
library(ggplot2)
library(paqueteMETODOS)

# Cargar datos
data(vivienda_faltantes)
vivienda <- vivienda_faltantes

# Análisis del precio por zona
resumen_precios <- vivienda %>%
  group_by(zona) %>%
  summarise(precio_promedio = mean(preciom, na.rm = TRUE), .groups = 'drop')

ggplot(resumen_precios, aes(x = zona, y = precio_promedio)) +
  geom_bar(stat = "identity") +
  theme_minimal() +
  labs(title = "Precio Promedio de Viviendas por Zona",
       x = "Zona",
       y = "Precio Promedio")
\end{verbatim}

\end{document}
